\section{Contextualização}

As bolsas de valores são infraestruturas de mercado onde compradores e vendedores negociam valores mobiliários sob regras claras e supervisão, promovendo formação de preços, liquidez e alocação eficiente de recursos \cite{MishkinEakins2018, FabozziModigliani2009}. No Brasil, esse papel é desempenhado pela B3 — Brasil, Bolsa, Balcão —, que integra ambientes de negociação, registro, compensação e liquidação, operando sob marcos legais como a Lei do Mercado de Valores Mobiliários e a Lei das Sociedades por Ações \cite{b3_quem_somos,Lei6385,Lei6404}. Em mercados acionários, os preços refletem expectativas sobre fluxos de caixa futuros e risco, mas também exibem forte volatilidade e variação de liquidez \cite{BodieKaneMarcus2021}. 

No horizonte intradiário, emergem problemas clássicos de microestrutura — periodicidades sistemáticas ao longo do dia, espaçamento irregular entre negócios e impacto de ordens (inclusive via \textit{spread}\footnote{Diferença entre o preço de compra (bid) e o de venda (ask) de um ativo.} e custo de imediatismo) — que condicionam a dinâmica de preços e a interpretação de sinais de curto prazo \cite{andersen1997intraday,engle1998acd,hasbrouck2007empirical,amihud2002illiquidity,dufour2000time}. Além disso, choques macroeconômicos e eventos exógenos podem deslocar rapidamente as cotações, elevando a incerteza \cite{WorldBank2020}. Esse contexto motiva a investigação de métodos preditivos sensíveis ao tempo e às nuances operacionais do ambiente de negociação brasileiro.

Com os avanços da tecnologia, a complexidade e a volatilidade dos ativos negociados em bolsa de valores aumentaram de forma significativa. Embora o mercado de ações não seja teoricamente previsível, muitos investidores identificam padrões que motivam a busca por estratégias de previsão\cite{lo2017}. No mundo, ativos fortemente correlacionados com mercados internacionais e fluxo de notícias, bem como a crescente participação de investidores individuais e institucionais, reforçam a necessidade de ferramentas que esclareçam sinais de curto, médio e longo prazo.

Neste estudo, considera-se explicitamente um modelo híbrido baseado em redes recorrentes do tipo LSTM combinadas a convoluções unidimensionais (CNN+LSTM), visando capturar simultaneamente dependências de longo prazo e padrões locais em séries intradiárias da B3. Essa escolha se apoia em evidências de que LSTMs representam melhor relações não lineares e dependências temporais prolongadas, enquanto camadas convolucionais extraem motivos de curto prazo a partir de janelas deslizantes; em conjunto, busca-se maior robustez frente a ruído e a mudanças de regime de mercado. Para assegurar comparabilidade, os resultados serão confrontados com baselines estatísticos (ARIMA, Prophet) e com variantes baseadas em Transformers reportadas nas literaturas \cite{ding2020, nascimento2023bwaif, zeng2022effectiveness, wen2022survey}.

No entanto, como Zvi Bodie and Alex Kane and Alan J. Marcus \cite{BodieKaneMarcus2021} dizem em seu livro, o mercado de ações é impulsionado por variáveis econômicas (taxas de juros, taxas de câmbio, indicadores de inflação), fatores corporativos (demonstrações financeiras da empresa e mudanças na governança) e também eventos externos (crises políticas, pandemias, choques de oferta) como diz \cite{WorldBank2020}. Estes quesitos permitem que, em cada série temporal, sejam produzidas componentes como tendência, sazonalidade e ciclos, conforme a decomposição tradicional proposta em Box e Jenkins \cite{box_2015} e amplamente discutida em Hyndman e Athanasopoulos \cite{hyndman_2018}:

\begin{itemize}
      \item Tendência: define se os preços atuais estão subindo ou caindo a curto prazo.
      \item Sazonalidade: é qualquer tipo de padrão repetitivo entre ciclos diários a anuais.
      \item Ciclos: um movimento de oscilação de médio a longo prazo, que não segue um calendário fixo, mas decorre de condições econômicas, corporativas e externas.
      \item Ruído aleatório: variações imprevisíveis, que não podem ser explicadas por tendência, sazonalidade ou ciclos, geralmente decorrentes de choques ou flutuações aleatórias.
\end{itemize}

A figura \ref{fig:series-temporais2} apresenta diferentes combinações dos componentes básicos de uma série temporal: tendência, ciclos e ruído. Cada gráfico mostra isoladamente ou em conjunto esses elementos, evidenciando como eles moldam o comportamento dos dados ao longo do tempo. A tendência representa o movimento de longo prazo (aumento ou queda contínua), os ciclos mostram padrões repetitivos de médio prazo ligados a fatores econômicos ou sazonais, e o ruído corresponde a flutuações aleatórias sem padrão definido. Assim, ao combinar esses elementos, observamos séries mais complexas e realistas, aproximando-se do que ocorre em dados empíricos.


\begin{figure}[H] % h=aqui, t=topo, b=base, p=página de floats, !=relaxa limites
  \centering
  \includegraphics[width=0.6\linewidth]{series-temporais2.png}
  \caption{Explicação visual sobre cada serie temporal. Fonte: \cite{limnonews2014}.}
  \label{fig:series-temporais2}
\end{figure}

É aqui que avanços arquitetônicos recentes entram em cena — como o \textit{Temporal Fusion Transformer} (TFT) \cite{lim2021temporal}, variantes Transformers voltadas à não-estacionaridade \cite{liu2022nonstationary}, mecanismos de atenção por “patches” (PatchTST) \cite{nie2023patchtst} e redes temporais dedicadas como o \textit{TimesNet} \cite{wu2023timesnet}. Além disso, modelos fundacionais e abordagens baseadas em \textbf{Grandes Modelos de Linguagem (LLMs)} vêm sendo aplicados a séries temporais, como o \textit{TimesFM} \cite{goel2025timesfm} e o \textit{TimeGPT} \cite{lathuilere2024timegpt}, juntamente com métodos que alinham LLMs à tarefa de previsão (LLM4TS; Time-LLM) \cite{chang2023llm4ts,jin2023time_llm} e aplicações no domínio financeiro \cite{noguer2024llm_financial}. Há também um debate quanto à efetividade prática dos Transformers em séries temporais \cite{zeng2022effectiveness,wen2022survey}. Para cenários com dados multimodais, destacam-se propostas como o \textit{Hidformer} \cite{liang2025hidformer}. Apesar desses avanços, contextos intradiários permanecem desafiadores — periodicidade intradiária e efeitos de microestrutura \cite{andersen1997intraday,hasbrouck2007empirical}, irregularidade temporal/duração entre negócios \cite{engle1998acd} e custo computacional — tema que será retomado na seção de Trabalhos Relacionados.


A própria natureza do comércio, marcada por negociações frequentes e pelo uso de estratégias algorítmicas — isto é, sistemas automatizados que executam ordens de compra e venda com base em regras predefinidas, muitas vezes em alta frequência —, enfatiza a importância de modelos capazes de capturar rapidamente padrões transitórios sem perder de vista o horizonte de médio prazo de maneira rigorosa \cite{sunki2024}.

Outras pesquisas já propuseram modelos híbridos para previsão de preços de ações — por exemplo, o modelo proposto por \cite{marcondes2013modelo}, que combina redes neurais, técnicas estatísticas e indicadores técnicos via comitê de aprendizado. Entretanto, tais propostas, em geral, não exploram esquemas de agregação mais sofisticados (como metaclassificação em dois estágios) nem avaliam de forma sistemática a robustez a mudanças súbitas de regime \cite{assis2022predicao,tsay2010}. 

Coerente com o título desta monografia, explicitamos os \textbf{alvos preditivos (indicativos financeiros)} a serem modelados: direção do return na próxima barra,  medida de volatilidade intradiária (p.ex., amplitude/variância realizada) e sinais de liquidez, como volume negociado e spread efetivo \cite{tsay2010,dacorogna2001,wood1985,MishkinEakins2018,rotela2014arima,nascimento2023bwaif}. 

Portanto, este trabalho avança a linha dos híbridos ao empregar um modelo CNN+LSTM com camadas convolucionais para padrões locais, validação temporal walk-forward e uma avaliação realista pós-custos no contexto da B3, com foco em estabilidade e reprodutibilidade dos resultados.




\section{Limitações dos modelos clássicos}

Métodos estatísticos tradicionais, como ARIMA e seus desdobramentos sazonais, oferecem boa interpretabilidade e desempenho em séries estacionárias ou de baixa não linearidade. Todavia, como demonstram estudos recentes, esses modelos tendem a perder eficácia quando confrontados com mercados “barulhentos”, em que pequenos choques(notícias inesperadas, decisões judiciais, compras ou vendas grandes em pouco tempo) podem alterar radicalmente as projeções \cite{sunki2024}. 

Ding e Qin \cite{ding2020} demonstraram o modelo associado de rede baseada em Long Short-Term Memory - LSTM\footnote{rede recorrente com memória de longo prazo} é capaz de predizer simultaneamente preço de abertura, mínima e máxima, alcançando acurácia superior a 95\% em dados históricos de bolsa. \cite{nascimento2023bwaif} compararam modelos ARIMA, Prophet e LSTM para séries financeiras, mostraram que arquiteturas baseadas em redes recorrentes capturam melhor dependências de tempo mais longas. Já propostas de \textit{ensembles} híbridos (por exemplo, um comitê que combina múltiplas LSTMs paralelas) sugerem ganhos de robustez e precisão. Diante do cenário dos modelos híbridos e \emph{ensembles}, que combinam redes profundas, métodos estatísticos e indicadores técnicos, a fim de aproveitar as vantagens de cada abordagem e mitigar suas fragilidades, tem-se o exemplo promissor que é o uso de metaclassificadores, em que vários algoritmos (e.g., redes LSTM, SVM, random forest) produzem predições independentes que são então combinadas por um modelo adicional, responsável por aprender como integrar essas saídas e gerar uma decisão final mais consistente \cite{assis2022predicao}.

Ao mesmo tempo, a inclusão de indicadores técnicos, como médias móveis, índice de força relativa (RSI) ou bandas de Bollinger, tem se mostrado útil para modelos de ML no enriquecimento de informações de tendência e volatilidade marcadas por janelas de preços históricos. Isso melhora a capacidade das redes neurais de discriminar entre sinais reais e ruído \cite{ding2020}. Se a volatilidade for suficientemente alta, esses comitês têm se mostrado mais estáveis e confiáveis na previsão do que qualquer um desses métodos usados isoladamente.

Embora vários estudos tenham afirmado precisões acima de 90\% no caso de previsões diárias, análises subsequentes mostraram um viés de "falsa segurança": muitos resultados não são muito úteis em cenários reais de mercado porque ignoram a dependência temporal real entre treinamento e teste.


\section{Problema de Pesquisa}

Resultados de previsão intradiária frequentemente não se sustentam em uso real porque violam a ordem temporal, ignoram regimes de liquidez/volatilidade e desconsideram custos de transação. Nesse cenário, permanece incerto se modelos de aprendizado profundo — em especial arquiteturas híbridas CNN+LSTM — geram sinais direcionais realmente utilizáveis (horizonte +1 barra/15 min) para ações líquidas da B3 entre 2020–2025, quando avaliados sob um protocolo rigoroso de validação walk-forward e comparados a baselines clássicos e arquiteturas com atenção.

\textbf{Delimitação do escopo:}

Este estudo situa-se no universo de bolsas de valores e mercados organizados, em que a negociação se dá sob regras de listagem, leilões, formação de preço e procedimentos operacionais definidos (no caso brasileiro, pela B3) \cite{b3_manual_negociacao}. Em janelas intradiárias, o comportamento do preço é influenciado por aspectos de \textit{microestrutura} (ex.: liquidez, \textit{spread}, intensidade de ordens, periodicidades intradiárias), o que afeta diretamente a previsibilidade e a avaliação de modelos \cite{hasbrouck2007empirical,andersen1997intraday,dufour2000time,engle1998acd}. À luz dessa dinâmica, explicitamos a seguir as delimitações adotadas.



\begin{itemize}
    \item \textbf{Ativos}: Consideramos ações líquidas negociadas na B3 (e.g., PETR4, VALE3, ITUB4) e, quando pertinente, a carteira do Ibovespa como referência de mercado. A escolha por alta liquidez visa mitigar efeitos de microestrutura adversos (como slippage e spread amplos) que distorcem sinais intradiários \cite{hasbrouck2007empirical}.
    
    \item \textbf{Período e granularidade}: Os experimentos cobrem jan/2020–jul/2025 em barras de 15 minutos, com análises de sensibilidade em 5 e 30 minutos. Essa granularidade equilibra ruído e sinal em alta frequência e captura periodicidades intradiárias conhecidas (picos de volatilidade na abertura/fechamento, padrões de almoço, etc.) \cite{andersen1997intraday,hyndman_2018}. 
    
    \item \textbf{Tarefa}: Prever a \textbf{direção do movimento de preço} na próxima vela (horizonte $+1$ barra), formulada como classificação probabilística. Tal formulação permite avaliar calibração e qualidade probabilística, não apenas acerto bruto \cite{brier1950,guo2017} — ver também a interpretação visual da tarefa na Figura~\ref{fig:delimitacao-tarefa}.

    \item \textbf{Validação e treinamento}: Para evitar viés temporal (ook-ahead) e superestimação de desempenho, adotamos validação walk-forward com janelas sequenciais, seleção de hiperparâmetros por otimização Bayesiana e controle de sementes/versionamento para reprodutibilidade. Esse arranjo segue boas práticas de validação em séries temporais e backtesting \cite{bergmeir2012,lopezdeprado2018}.

\item \textbf{Métricas}: Aferimos: 
\begin{enumerate}[label=\Roman*.]
    \item \emph{acurácia direcional com banda morta}, para robustez a microvariações;
    \item Brier Score e Log-Loss, para qualidade das probabilidades e calibração \cite{brier1950,guo2017};
    \item métricas pós-custos de trading (retorno líquido, drawdown e índice de Sharpe), refletindo utilidade prática sob custos de transação \cite{sharpe1994}
\end{enumerate}
Para comparar previsores quando necessário, usamos testes de acurácia preditiva como Diebold–Mariano \cite{diebold1995}.


\item \textbf{Comparadores (baselines)}: Contrastamos o modelo proposto com referências clássicas e modernas: 
\begin{enumerate}[label=\Roman*.]
    \item ARIMA/Box–Jenkins como baseline estatístico canônico \cite{box_2015,hyndman_2018,rotela2014arima}; 
    
    \item Prophet, representando um decompositor aditivo com sazonalidades/mudanças de nível \cite{taylor_2018};
    
    \item arquiteturas modernas para séries temporais, incluindo variantes baseadas em Transformers, dada sua capacidade de capturar dependências de longo alcance \cite{wen2022survey,zeng2022effectiveness}. 
\end{enumerate}

\end{itemize}


A ilustração \ref{fig:delimitacao-tarefa} sintetiza a tarefa: em uma série de preços intradiária (barras de 15 min), cada ponto no tempo gera um vetor de atributos (preço, volume, indicadores técnicos e variáveis de contexto) e o modelo estima a probabilidade de a próxima barra encerrar acima (ou abaixo) da atual. O horizonte é de +1 barra (curto prazo), e as janelas de treino/teste avançam sequencialmente no tempo (esquema walk-forward), refletindo o fluxo real de mercado e evitando contaminação temporal.


\begin{figure}[H] % h=aqui, t=topo, b=base, p=página de floats, !=relaxa limites
  \centering
  \includegraphics[width=0.6\linewidth]{delimitacao-tarefa.png}
  \caption{Delimitação da tarefa: previsão direcional intradiária (+1 barra, 15 min). Fonte: autor.}
  \label{fig:delimitacao-tarefa}
\end{figure}



\textbf{Pergunta central do estudo:}

\begin{quote}
Em que condições (janelas, regimes de volatilidade e custo) um modelo CNN+LSTM supera comparadores clássicos (Naive, ARIMA, Prophet) e variantes com atenção na tarefa de previsão direcional intradiária (+1 barra/15 min) em ações líquidas da B3 (2020–2025), sob avaliação walk-forward e métricas pós-custos?

\end{quote}



\subsection{Justificativa}
A previsão intradiária de movimentos de preços na B3 apresenta relevância prática para apoio à decisão em ambientes voláteis e com custos de transação não desprezíveis. Do ponto de vista acadêmico, persiste uma lacuna entre resultados reportados em ambientes controlados e o desempenho sob \textit{walk-forward}\footnote{Método em que o treino e o teste avançam em janelas sequenciais, mantendo a ordem temporal.}, com avaliação pós-custos e controle de deriva de conceito. Este estudo busca reduzir essa lacuna ao (i) delimitar um cenário intradiário realista, (ii) empregar validação temporal adequada e (iii) comparar um modelo híbrido (CNN+LSTM) a \textit{baselines} consolidados. O potencial impacto inclui melhores critérios de avaliação para tarefas direcionais, evidências sobre robustez em diferentes regimes de volatilidade e um \textit{pipeline} reprodutível que pode ser replicado e estendido por outros pesquisadores.



\section{Objetivos}

\subsection{Objetivo geral}
Avaliar de forma quantitativa e reprodutível, a capacidade de modelos de aprendizado de máquina sensíveis ao tempo em prever o comportamento direcional de indicadores financeiros intradiários, comparando seu desempenho a baselines estatísticos e neurais sob um protocolo experimental realista (validação walk-forward e backtext\footnote{Simulação histórica de uma estratégia sob regras e dados passados.}) pós-custos.

\subsection{Objetivos específicos}
\begin{enumerate}
  \item Construir e documentar o conjunto de dados intradiários (OHLCV e indicadores técnicos) para o período definido (2020--2025) e frequência fixa (15 minutos), especificando fontes, regras de limpeza, critérios de inclusão de ativos e prevenção de vazamento temporal.
  \item Definir o desenho experimental com validação walk-forward (janelas deslizantes, embargo temporal e padronização dentro de cada janela), estratégia de seleção de hiperparâmetros e controle de variáveis (sementes, versões e ambiente).
  \item Treinar e comparar \textit{baselines} representativos com o(s) modelo(s) proposto(s) sensíveis à ordem temporal, mantendo critérios comparáveis de otimização.
  \item Mensurar a qualidade preditiva por múltiplas métricas: acurácia direcional com banda morta, F1/recall para a classe positiva, Brier score e AUC-PR, incluindo análise de calibração das probabilidades.
  \item Avaliar desempenho operacional convertendo sinais em regras simples de negociação e realizando backtests com custos de transação e slippage, reportando retorno, Sharpe, max drawdown e turnover, além de cortes por regimes de volatilidade.
  \item Testar significância e robustez dos resultados por meio do teste de Diebold-Mariano, bootstrap em blocos e análises de sensibilidade (janelas temporais, subconjuntos de ativos, horizonte de previsão e feature sets).
  \item Garantir reprodutibilidade com repositório versionado, manifesto de dependências, \textit{scripts} de preparo/treino/avaliação, diários de experimento e registro explícito de ameaças à validade interna e externa.
\end{enumerate}




\subsection{Contribuição Esperada}


Este trabalho busca contribuir em quatro frentes principais: 

Primeiro, com o desenvolvimento de um modelo híbrido (CNN+LSTM) capaz de combinar padrões locais e dependências de longo prazo em séries temporais financeiras, adequando-se às particularidades do mercado intradiário da B3. 

Em segundo lugar, ao realizar uma avaliação empírica comparativa com \textit{baselines} estabelecidos (ARIMA, Prophet, LSTM puro e variantes Transformer), sob um protocolo realista de validação walk-forward e backtests pós-custos. 

A terceira contribuição está na análise da robustez do modelo em diferentes condições de mercado, explorando regimes de volatilidade e choques pontuais. 

Finalmente, planeja-se lançar um pipeline reprodutível com a fonte organizada, bem como documentação detalhada, não apenas para apoiar a futura replicação de experimentos, mas também para facilitar a adaptação e extensão por outros pesquisadores interessados nessas abordagens.

Este trabalho promoverá e eliminará barreiras técnicas à comunidade científica e, desencadeará a co-construção do conhecimento. Além disso, a natureza aberta do pipeline permite que novas metodologias sejam testadas em igualdade de condições, tornando os resultados mais comparáveis e, portanto, contribuindo para conclusões mais robustas.
