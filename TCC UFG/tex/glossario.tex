

\begin{itemize}

\item{Gap} — Espaço no gráfico de preços entre dois períodos consecutivos, causado por variação abrupta na cotação.
\item{Slippage} — Diferença entre o preço esperado e o preço efetivo de execução de uma ordem.
\item{Spread} — Diferença entre o preço de compra (bid) e o de venda (ask) de um ativo.
\item{Survivorship bias} — Viés de análise que considera apenas casos que “sobreviveram” a um processo, ignorando os demais.
\item{Splits} – No contexto de mercados financeiros, refere-se à divisão ou agrupamento das ações de uma empresa, alterando o número de papéis em circulação sem modificar o valor total investido. 
\begin{itemize}
    \item {Stock Split (desdobramento):} aumenta a quantidade de ações e reduz proporcionalmente o preço unitário.
    \item {Reverse Split (grupamento) ou Inplits:} reduz a quantidade de ações e eleva proporcionalmente o preço unitário.
\end{itemize}
\item{Time Zone} – No contexto do mercado financeiro, refere-se ao fuso horário adotado para registrar e sincronizar horários de negociações, cotações e eventos relevantes. A correta configuração do \textit{time zone} é essencial para alinhar dados de diferentes bolsas e mercados globais, garantindo precisão na análise temporal e na execução de ordens.
\item{Candlestick} – Tipo de gráfico utilizado no mercado financeiro que representa, por meio de velas, a abertura, fechamento, máxima e mínima dos preços de um ativo em um intervalo de tempo definido.
\item{Candles} – Representações gráficas, no formato de velas (\textit{candlestick}), que mostram abertura, fechamento, máxima e mínima de preços de um ativo em um intervalo de tempo no mercado financeiro.
\item{Spikes Espúrios} – No mercado financeiro, referem-se a variações abruptas e atípicas nos preços de um ativo, geralmente causadas por erros de dados, baixa liquidez ou eventos não representativos das condições reais de mercado.
\item{Tick-by-Tick} – No mercado financeiro, refere-se a dados que registram cada alteração de preço e volume negociado de um ativo, fornecendo o histórico completo das negociações em ordem cronológica.
\item{Timestamping} – No mercado financeiro, é o processo de registrar com precisão o instante exato em que uma negociação, evento ou dado de mercado ocorre, permitindo análises temporais e sincronização de informações.
\item{APIs} – No contexto de mercado financeiro e IA, são interfaces de programação que permitem a integração e troca de dados entre sistemas, possibilitando o acesso automatizado a cotações, históricos, ordens e outros serviços em tempo real.
\item{Min-Max Scaler} – No contexto de IA aplicada ao mercado financeiro, é uma técnica de normalização que transforma os dados para um intervalo pré-definido, geralmente entre 0 e 1, preservando as proporções originais. É usada para padronizar variáveis, como preços e indicadores, antes do treinamento de modelos.
\item{Z-Score} – No contexto de IA e mercado financeiro, é uma medida de padronização que indica quantos desvios-padrão um valor está distante da média. É utilizada para identificar anomalias, como preços atípicos ou variações incomuns em séries temporais.
\item{Walk-Forward} – No contexto de IA e mercado financeiro, é uma técnica de validação em séries temporais que consiste em treinar um modelo em um intervalo de dados e testá-lo no período subsequente, avançando progressivamente no tempo para avaliar seu desempenho de forma contínua.
\item{Data Leakage} – No mercado financeiro, é o uso indevido de dados futuros no treinamento de modelos de IA, causando previsões artificialmente precisas e sem validade real.
\item{Imputation} – No mercado financeiro e IA, é o processo de preencher valores ausentes em séries temporais, garantindo consistência nos dados para análise ou treinamento de modelos.
\item{OHLCV} – No mercado financeiro, sigla para \textit{Open}, \textit{High}, \textit{Low}, \textit{Close} e \textit{Volume}, representando respectivamente abertura, máxima, mínima, fechamento e volume negociado de um ativo em um intervalo de tempo.
\item{Bayesiana} – No contexto de IA e mercado financeiro, refere-se a métodos baseados no Teorema de Bayes, que atualizam probabilidades à medida que novas informações se tornam disponíveis, auxiliando na previsão e na tomada de decisão.
\item{Dropout} – Em IA aplicada ao mercado financeiro, é uma técnica de regularização em redes neurais que desativa aleatoriamente neurônios durante o treinamento, reduzindo overfitting.
\item{Learning Rate} – Em IA aplicada ao mercado financeiro, é o parâmetro que define o tamanho dos ajustes nos pesos da rede neural a cada iteração de treinamento.
\item{Early Stopping} – Em IA para mercado financeiro, técnica que interrompe o treinamento ao detectar perda de desempenho em validação, evitando overfitting.
\item{Gradient Clipping} – Em IA para mercado financeiro, técnica que limita o valor máximo dos gradientes no treinamento, evitando instabilidade e divergência do modelo.
\item{Scheduler} – Em IA para mercado financeiro, mecanismo que ajusta automaticamente a \textit{learning rate} durante o treinamento para melhorar a convergência do modelo.
\item{Binary Cross-Entropy} – Em IA para mercado financeiro, função de perda usada em classificações binárias, medindo a diferença entre probabilidades previstas e valores reais.
\item{Drawdown} – No mercado financeiro, é a redução percentual do valor de um ativo ou portfólio desde um pico até o menor ponto subsequente.

\item{backtest} – Simulação histórica de uma estratégia sob regras e dados passados.

\item {Balanced Accuracy} - Média de sensibilidade e especificidade; robusta a classes desbalanceadas.
\item {Banda morta} - Zona neutra em torno de um limiar para ignorar microvariações e ruídos de microestrutura.
\item {Bandas de Bollinger} - Envelope ao redor de uma média móvel definido por desvios-padrão, sinalizando volatilidade e possíveis suportes/resistências.
\item {Acurácia direcional (hit rate)} - Proporção de acertos na previsão da direção (alta/baixa) do próximo movimento.
\item {Bid–ask bounce} - Oscilação do preço entre bid e ask sem mudança no valor justo, gerando ruído.
\item {Bootstrap em blocos} - Reamostragem por blocos contíguos para preservar dependência temporal em séries.
\item {Brier Score} - Erro quadrático médio entre probabilidades previstas e resultados binários observados.
\item {Co-location} - Hospedagem de servidores no data center da bolsa para reduzir latência de comunicação.
\item {Custos de transação} - Despesas e fricções (corretagem, emolumentos, spread, slippage) consideradas nos backtests.
\item {Embargo temporal} - Janela de exclusão após o treino para impedir contaminação temporal da validação/teste.
\item {F1-Score} - Média harmônica entre precisão (precision) e revocação (recall).
\item {Latência} - Atraso entre envio/recebimento de dados/ordens e sua efetiva execução/registro.
\item {Log-Loss} - Perda logarítmica usada em classificadores probabilísticos; penaliza confiança indevida.
\item {LSTM (conceito)} - Rede recorrente com memória de longo prazo para dependências temporais extensas.
\item {Microestrutura de mercado} - Aspectos de formação de preço em alta frequência (livro de ofertas, tipos de ordens, latência).
\item {Otimização Bayesiana} - Busca de hiperparâmetros guiada por modelo probabilístico do desempenho.
\item {Prophet} - Modelo aditivo com tendências e sazonalidades marcado para séries temporais.
\item {Regimes de volatilidade} - Estratificação do tempo por níveis de volatilidade para análises de robustez.
\item {Seed (semente aleatória)} - Estado inicial do gerador pseudoaleatório para reprodutibilidade.
\item {Teste de Diebold–Mariano} - Teste estatístico para comparar a acurácia preditiva de dois modelos.
\item {Transformers} - Arquitetura baseada em atenção para dependências de longo alcance sem recorrência.
\item {Turnover} - Taxa de giro da carteira (volume negociado relativo ao capital) em um período.
\item {Validação walk-forward} - Treina e testa em janelas deslizantes preservando a ordem temporal.
\item {Volatilidade} - Grau de oscilação dos preços em um período; associada à dispersão dos retornos.
\item {Índice de Sharpe} - Retorno excedente por unidade de risco (desvio-padrão dos retornos).




\end{itemize}