% ============================================================================
% Capítulo 5 — Resultados e Discussão
% (O \chapter é definido em thesis.tex; aqui só o conteúdo.)
% ============================================================================

Este capítulo apresenta os resultados obtidos com o modelo proposto (CNN--LSTM) e com os
\emph{baselines} definidos na metodologia. A análise segue o protocolo \emph{walk-forward}, com
embargo entre treino e teste, e usa como referência principal a acurácia direcional após a banda
morta de 0,1\%. Além disso, os resultados são complementados por testes de significância
(Diebold--Mariano) e por \emph{backtests} com custos explícitos.

Os experimentos foram executados para três ativos de alta liquidez (PETR4, VALE3 e ITUB4),
com barras intradiárias de 15 minutos. Em cada ativo, a avaliação é feita por \emph{fold} e, ao final,
reportamos o desempenho médio entre os \emph{folds}. Essa estrutura é importante porque aproxima
o cenário de um uso real: o modelo sempre prevê o “futuro” a partir do “passado”.

% ---------------------------------------------------------------------------
\section{Visão geral dos resultados e organização dos artefatos}
% ---------------------------------------------------------------------------

Os resultados consolidados foram organizados em quatro grupos de artefatos:

\begin{itemize}
  \item \textbf{Comparativo entre modelos}: arquivo
  \texttt{data/processed/comparativo\_cnn\_lstm\_vs\_baselines.csv}, com métricas médias por ativo e por modelo.
  \item \textbf{Resultados por fold}: arquivos \texttt{<ATIVO>\_cnn\_lstm\_walkforward.csv}, com métricas do CNN--LSTM ao longo do \emph{walk-forward}.
  \item \textbf{Teste de Diebold--Mariano}: arquivos \texttt{dm\_resumo\_pvalores.csv} e
  \texttt{testes\_diebold\_mariano.csv}, incluindo versão segmentada por regime de volatilidade.
  \item \textbf{Backtests}: arquivo \texttt{data/backtest/historico\_backtest.csv} (consolidado) e CSVs por ativo/fold/estratégia.
\end{itemize}

A leitura dos resultados neste capítulo segue essa mesma lógica: primeiro comparamos modelos
em termos de acerto estatístico, depois verificamos se a diferença é significativa e, por fim,
traduzimos isso em impacto financeiro via \emph{backtest}.

% ---------------------------------------------------------------------------
\section{Comparação geral: CNN--LSTM vs. baselines}
% ---------------------------------------------------------------------------

A Tabela~\ref{tab:comparativo-geral} resume o desempenho médio do CNN--LSTM e dos principais
\emph{baselines} por ativo. O foco está na acurácia direcional (após banda morta), mas também
reportamos métricas que ajudam a interpretar cenários desbalanceados e qualidade probabilística
(Brier e Log-Loss).

\begin{table}[htbp]
  \centering
  \caption{Comparativo médio entre CNN--LSTM e baselines (média dos folds; após banda morta).}
  \label{tab:comparativo-geral}
  \begin{tabular}{l l r r r r}
    \toprule
    \textbf{Ativo} & \textbf{Modelo} & \textbf{Hit rate} & \textbf{Bal. Acc.} & \textbf{MCC} & \textbf{Brier} \\
    \midrule
    PETR4 & CNN--LSTM & <<\%>> & <<>> & <<>> & <<>> \\
    PETR4 & Naive     & <<\%>> & <<>> & <<>> & <<>> \\
    PETR4 & Drift     & <<\%>> & <<>> & <<>> & <<>> \\
    PETR4 & ARIMA     & <<\%>> & <<>> & <<>> & <<>> \\
    PETR4 & Prophet   & <<\%>> & <<>> & <<>> & <<>> \\
    \midrule
    VALE3 & CNN--LSTM & <<\%>> & <<>> & <<>> & <<>> \\
    VALE3 & Naive     & <<\%>> & <<>> & <<>> & <<>> \\
    VALE3 & Drift     & <<\%>> & <<>> & <<>> & <<>> \\
    VALE3 & ARIMA     & <<\%>> & <<>> & <<>> & <<>> \\
    VALE3 & Prophet   & <<\%>> & <<>> & <<>> & <<>> \\
    \midrule
    ITUB4 & CNN--LSTM & <<\%>> & <<>> & <<>> & <<>> \\
    ITUB4 & Naive     & <<\%>> & <<>> & <<>> & <<>> \\
    ITUB4 & Drift     & <<\%>> & <<>> & <<>> & <<>> \\
    ITUB4 & ARIMA     & <<\%>> & <<>> & <<>> & <<>> \\
    ITUB4 & Prophet   & <<\%>> & <<>> & <<>> & <<>> \\
    \bottomrule
  \end{tabular}
  \fonte{O autor, a partir de \texttt{comparativo\_cnn\_lstm\_vs\_baselines.csv}.}
\end{table}

De forma geral, essa tabela responde a primeira pergunta do trabalho: \emph{o modelo proposto
traz ganho real sobre métodos simples?} A comparação é importante porque, em finanças, ganhos
pequenos podem ser relevantes. Ao mesmo tempo, eles precisam ser consistentes. Por isso, a leitura
não deve ficar apenas no “melhor valor”. Ela precisa considerar estabilidade e significância.

Também é útil observar as métricas probabilísticas. Em especial, o Brier e o Log-Loss ajudam a
entender se as probabilidades do modelo têm “qualidade”, e não apenas aparência de número. Isso
importa porque o \emph{backtest} depende de limiares e decisões de entrada/saída.

% ---------------------------------------------------------------------------
\section{Estabilidade no tempo: desempenho por fold do walk-forward}
% ---------------------------------------------------------------------------

A média é um bom resumo, mas não conta a história inteira. Em séries financeiras,
o desempenho pode variar bastante de um período para outro. Por isso, analisamos os resultados
por \emph{fold}. Isso permite enxergar duas coisas: (i) se o ganho do CNN--LSTM é recorrente e
(ii) se existem períodos em que o modelo degrada de forma mais forte.

A Tabela~\ref{tab:folds-cnnlstm} apresenta um exemplo de como organizar os resultados do
CNN--LSTM por \emph{fold}. Aqui, a ideia é simples: mostrar a variabilidade e comentar os extremos.

\begin{table}[htbp]
  \centering
  \caption{CNN--LSTM por fold (exemplo de preenchimento a partir de \texttt{<ATIVO>\_cnn\_lstm\_walkforward.csv}).}
  \label{tab:folds-cnnlstm}
  \begin{tabular}{l r r r r}
    \toprule
    \textbf{Ativo} & \textbf{Fold} & \textbf{Hit rate} & \textbf{MCC} & \textbf{Brier} \\
    \midrule
    PETR4 & 1 & <<>> & <<>> & <<>> \\
    PETR4 & 2 & <<>> & <<>> & <<>> \\
    PETR4 & 3 & <<>> & <<>> & <<>> \\
    PETR4 & 4 & <<>> & <<>> & <<>> \\
    PETR4 & 5 & <<>> & <<>> & <<>> \\
    \midrule
    VALE3 & 1 & <<>> & <<>> & <<>> \\
    VALE3 & 2 & <<>> & <<>> & <<>> \\
    VALE3 & 3 & <<>> & <<>> & <<>> \\
    VALE3 & 4 & <<>> & <<>> & <<>> \\
    VALE3 & 5 & <<>> & <<>> & <<>> \\
    \midrule
    ITUB4 & 1 & <<>> & <<>> & <<>> \\
    ITUB4 & 2 & <<>> & <<>> & <<>> \\
    ITUB4 & 3 & <<>> & <<>> & <<>> \\
    ITUB4 & 4 & <<>> & <<>> & <<>> \\
    ITUB4 & 5 & <<>> & <<>> & <<>> \\
    \bottomrule
  \end{tabular}
  \fonte{O autor.}
\end{table}

Na discussão, vale observar se existem \emph{folds} em que o desempenho cai muito. Quando isso
acontece, a causa mais comum é mudança de regime. Pode ser um período mais volátil. Pode ser um
período com tendência forte. Ou pode ser um momento em que o preço “oscila sem direção”.
Esses cenários são importantes porque eles expõem o limite prático do modelo.

% ---------------------------------------------------------------------------
\section{Top configurações do Optuna e sensibilidade do lookback}
% ---------------------------------------------------------------------------

Além do desempenho final, analisamos as melhores configurações encontradas na busca bayesiana.
Esse ponto ajuda a responder outra pergunta: \emph{o modelo depende de um ajuste “muito fino” para funcionar?}
Se a resposta for “sim”, isso aumenta o risco de instabilidade fora da amostra.

A Tabela~\ref{tab:top10-optuna} é um modelo de como reportar o “top 10” do tuning
(um resumo por ativo costuma ser suficiente no texto principal; o restante pode ir para apêndice).

\begin{table}[htbp]
  \centering
  \caption{Top configurações (Optuna) para o CNN--LSTM (exemplo; preencher a partir de \texttt{<ATIVO>\_cnn\_lstm\_analise\_modelos.csv}).}
  \label{tab:top10-optuna}
  \begin{tabular}{r r r r r r}
    \toprule
    \textbf{Rank} & \textbf{Filtros} & \textbf{Kernel} & \textbf{LSTM} & \textbf{Dropout} & \textbf{Hit rate} \\
    \midrule
    1  & <<>> & <<>> & <<>> & <<>> & <<>> \\
    2  & <<>> & <<>> & <<>> & <<>> & <<>> \\
    3  & <<>> & <<>> & <<>> & <<>> & <<>> \\
    4  & <<>> & <<>> & <<>> & <<>> & <<>> \\
    5  & <<>> & <<>> & <<>> & <<>> & <<>> \\
    6  & <<>> & <<>> & <<>> & <<>> & <<>> \\
    7  & <<>> & <<>> & <<>> & <<>> & <<>> \\
    8  & <<>> & <<>> & <<>> & <<>> & <<>> \\
    9  & <<>> & <<>> & <<>> & <<>> & <<>> \\
    10 & <<>> & <<>> & <<>> & <<>> & <<>> \\
    \bottomrule
  \end{tabular}
  \fonte{O autor.}
\end{table}

Aqui vale comentar também o \textbf{lookback}. Como a janela padrão é 60, mas janelas 16/32/64
foram avaliadas, a discussão pode ser direta: se o desempenho muda pouco, isso é um bom sinal.
Significa que o modelo não está preso a um único horizonte. Se muda muito, isso indica sensibilidade.
Nesse caso, a escolha do lookback vira uma decisão crítica do pipeline.

% ---------------------------------------------------------------------------
\section{Significância estatística: teste de Diebold--Mariano}
% ---------------------------------------------------------------------------

A comparação por métricas é útil, mas ainda falta uma peça. Em finanças, diferenças pequenas
podem acontecer “por sorte”, principalmente em janelas curtas. Para reduzir esse risco, aplicamos
o teste de Diebold--Mariano (DM), comparando a série de perdas do CNN--LSTM com a de cada baseline.

A Tabela~\ref{tab:dm-resumo} organiza os p-valores do DM (versão geral). A interpretação é simples:
p-valores baixos indicam evidência de que a diferença média de desempenho não é aleatória.
Na prática, isso ajuda a sustentar a conclusão de que o modelo proposto realmente melhora o baseline.

\begin{table}[htbp]
  \centering
  \caption{Resumo do teste de Diebold--Mariano (p-valores; preencher a partir de \texttt{dm\_resumo\_pvalores.csv}).}
  \label{tab:dm-resumo}
  \begin{tabular}{l l r}
    \toprule
    \textbf{Ativo} & \textbf{Comparação} & \textbf{p-valor} \\
    \midrule
    PETR4 & CNN--LSTM vs Naive   & <<>> \\
    PETR4 & CNN--LSTM vs Drift   & <<>> \\
    PETR4 & CNN--LSTM vs ARIMA   & <<>> \\
    PETR4 & CNN--LSTM vs Prophet & <<>> \\
    \midrule
    VALE3 & CNN--LSTM vs Naive   & <<>> \\
    VALE3 & CNN--LSTM vs Drift   & <<>> \\
    VALE3 & CNN--LSTM vs ARIMA   & <<>> \\
    VALE3 & CNN--LSTM vs Prophet & <<>> \\
    \midrule
    ITUB4 & CNN--LSTM vs Naive   & <<>> \\
    ITUB4 & CNN--LSTM vs Drift   & <<>> \\
    ITUB4 & CNN--LSTM vs ARIMA   & <<>> \\
    ITUB4 & CNN--LSTM vs Prophet & <<>> \\
    \bottomrule
  \end{tabular}
  \fonte{O autor.}
\end{table}

Se você preferir, dá para incluir também o \textbf{heatmap} de p-valores como figura.
Ele costuma ficar bem visual no relatório:

\begin{figure}[H]
  \centering
  \includegraphics[width=0.9\linewidth]{dm_heatmap_pvalores.png}
  \caption{Heatmap de p-valores do teste de Diebold--Mariano (comparação CNN--LSTM vs baselines). Fonte: o autor.}
  \label{fig:dm-heatmap}
\end{figure}

Na discussão, vale tomar cuidado com dois pontos. Primeiro: significância não é “garantia de lucro”.
Ela apenas diz que a diferença estatística é consistente. Segundo: se o DM for significativo apenas
em um ativo, isso ainda é um resultado válido. Só muda a força da generalização.

% ---------------------------------------------------------------------------
\section{Backtests: impacto prático após custos}
% ---------------------------------------------------------------------------

Depois de medir acerto e significância, a pergunta final é a mais prática:
\emph{o modelo gera resultado financeiro quando custos entram na conta?}
Para isso, executamos \emph{backtests} em duas modalidades: long-only e long/short.

A Tabela~\ref{tab:backtest-longonly} resume long-only. Ela é útil porque é o cenário mais simples.
O sinal de baixa vira “ficar em caixa”. Isso costuma reduzir risco e também reduz o número de operações.

\begin{table}[htbp]
  \centering
  \caption{Backtest long-only (após custos; preencher a partir de \texttt{historico\_backtest.csv}).}
  \label{tab:backtest-longonly}
  \begin{tabular}{l r r r r}
    \toprule
    \textbf{Ativo} & \textbf{Retorno líquido} & \textbf{Sharpe} & \textbf{Max DD} & \textbf{Profit factor} \\
    \midrule
    PETR4 & <<>> & <<>> & <<>> & <<>> \\
    VALE3 & <<>> & <<>> & <<>> & <<>> \\
    ITUB4 & <<>> & <<>> & <<>> & <<>> \\
    \bottomrule
  \end{tabular}
  \fonte{O autor.}
\end{table}

A Tabela~\ref{tab:backtest-longshort} resume long/short. Esse modo é mais agressivo.
Ele pode ganhar em tendências de queda, mas normalmente “paga” com mais giro e mais custo.
Por isso, o que importa aqui é observar retorno \emph{e} risco (Sharpe e drawdown).

\begin{table}[htbp]
  \centering
  \caption{Backtest long/short (após custos; preencher a partir de \texttt{historico\_backtest.csv}).}
  \label{tab:backtest-longshort}
  \begin{tabular}{l r r r r}
    \toprule
    \textbf{Ativo} & \textbf{Retorno líquido} & \textbf{Sharpe} & \textbf{Max DD} & \textbf{Profit factor} \\
    \midrule
    PETR4 & <<>> & <<>> & <<>> & <<>> \\
    VALE3 & <<>> & <<>> & <<>> & <<>> \\
    ITUB4 & <<>> & <<>> & <<>> & <<>> \\
    \bottomrule
  \end{tabular}
  \fonte{O autor.}
\end{table}

Na discussão, é importante deixar claro o seguinte: um modelo pode ter bom \emph{hit rate} e ainda
assim não ser lucrativo. Isso acontece quando o ganho médio por operação é pequeno e o custo come
a margem. Por isso, turnover e sensibilidade a custos entram como parte essencial da análise.

% ---------------------------------------------------------------------------
\section{Robustez: regimes de volatilidade}
% ---------------------------------------------------------------------------

Por fim, segmentamos a avaliação por regimes de volatilidade. A motivação é direta:
o mercado não se comporta da mesma forma o tempo todo. Em períodos calmos, padrões técnicos
podem aparecer com mais clareza. Em choques, o ruído aumenta e o preço pode mudar de regime
rapidamente.

A análise separa as observações em baixa e alta volatilidade (em relação à mediana), e aplica o
teste de Diebold--Mariano em cada segmento. Os resultados podem ser reportados de forma resumida
na Tabela~\ref{tab:dm-regimes}. Ela ajuda a responder se o ganho do CNN--LSTM é “geral” ou se ele
depende mais do tipo de mercado.

\begin{table}[htbp]
  \centering
  \caption{Teste DM por regime de volatilidade (p-valores; preencher a partir de \texttt{dm\_resumo\_pvalores\_regime\_*.csv}).}
  \label{tab:dm-regimes}
  \begin{tabular}{l l r r}
    \toprule
    \textbf{Ativo} & \textbf{Comparação} & \textbf{p (baixa vol)} & \textbf{p (alta vol)} \\
    \midrule
    PETR4 & CNN--LSTM vs baselines & <<>> & <<>> \\
    VALE3 & CNN--LSTM vs baselines & <<>> & <<>> \\
    ITUB4 & CNN--LSTM vs baselines & <<>> & <<>> \\
    \bottomrule
  \end{tabular}
  \fonte{O autor.}
\end{table}

Se o modelo mantém desempenho no regime de alta volatilidade, isso é um sinal forte de robustez.
Se ele cai muito, isso não “invalida” o método. Mas mostra uma limitação prática clara. Nessa situação,
uma extensão natural do trabalho é adaptar limiares por regime, ou introduzir um detector de regime
para ajustar a agressividade da estratégia.

% ---------------------------------------------------------------------------
\section{Síntese}
% ---------------------------------------------------------------------------

Em resumo, os resultados deste capítulo são organizados em três níveis.
No primeiro nível, verificamos se o CNN--LSTM supera baselines em métricas de classificação.
No segundo, avaliamos se a diferença é estatisticamente consistente via Diebold--Mariano.
No terceiro, testamos utilidade prática com \emph{backtests} após custos.

Essa estrutura evita uma conclusão baseada em um único número. Ela força o resultado a “passar”
por filtros diferentes. E isso é especialmente importante em finanças, onde instabilidade é a regra,
não a exceção.
