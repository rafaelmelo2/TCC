% ------------------------------------------------------------------------
% UFGRC: Modelo de Trabalho Acadêmico em conformidade com 
% ABNT NBR 14724:2011
% ------------------------------------------------------------------------
\PassOptionsToPackage{acronym}{glossaries}
\documentclass[tcc1, pos-defesa, english, brazil]{packages/ufgrc}

% Pacotes adicionais
\usepackage{rotating}
\usepackage[all,knot,arc,import,poly]{xy}
\graphicspath{{../images/}}

% \hypersetup{colorlinks=true}Q
\usepackage{tabularx,booktabs,array}
\usepackage{float} % para usar [H]
\newcolumntype{Y}{>{\raggedright\arraybackslash}X} % coluna com quebra automática

% Dados da capa e folha de rosto
\titulo{Predição Automática de Indicativos Financeiros para Bolsa de Valores Considerando o Aspecto Temporal}
\autor[Melo, R. S.]{Rafael da Silva Melo}
\orientador[Orientador]{Prof. Dr.}{Márcio de Souza Dias}
\genero{M}
\data{01}{08}{2025}
\membrobanca{Marcio Antonio Duarte}{Universidade Federal de Catalão}
\membrobanca{Thiago Jabur Bittar}{Universidade Federal de Catalão}

% Resumos
\textoresumo[brazil]{


O trabalho introduz um modelo híbrido de aprendizado profundo, utilizando uma combinação de redes neurais LSTM + CNN para prever indicadores financeiros da bolsa brasileira B3 — Brasil, Bolsa, Balcão — com base em dados de séries temporais intradiárias. 

A análise incorpora dados de liquidez, volatilidade e volume negociado, juntamente com engenharia de características e uma metodologia de validação cruzada temporalmente consciente para melhorar a robustez e a repetibilidade dos resultados.

Sua arquitetura foi construída para inferir cenários de estresse e choques de mercado, algo fora do alcance de métodos clássicos como ARIMA ou Prophet. Um importante marco é que produz valores muito precisos, o que importa muito mais para negociantes da bolsa e gerentes de risco.

Um pipeline baseado em Python será construído para ser usado na reprodução e geração de dados, além da agregação de dados e análise subsequente na qual este estudo se baseia. Essas contribuições implicam em uma maior estabilidade em ambientes voláteis e insights que podem ser transferidos para vários ativos financeiros.
}{predição, indicadores financeiros, séries temporais, bolsa de valores, aprendizado profundo}

\textoresumo[english]{
The paper introduces a hybrid deep learning model that combines LSTM and CNN neural networks to predict financial indicators of the Brazilian exchange B3 — Brasil, Bolsa, Balcão — based on intraday time-series data.

The analysis incorporates liquidity, volatility, and traded volume data, together with feature engineering and a time-aware cross-validation methodology to improve the robustness and replicability of the results.

Its architecture was designed to infer stress scenarios and market shocks, something beyond the reach of classical methods such as ARIMA or Prophet. A key advantage is that it produces highly accurate values, which matters much more to stock traders and risk managers.

A Python-based pipeline will be built to support reproducibility and data generation, as well as data aggregation and the subsequent analysis on which this study relies. These contributions imply greater stability in volatile environments and insights that can be transferred to multiple financial assets.
}{prediction, financial indicators, time series, stock exchange, deep learning}

% Inicio do documento
% (Capa, folhas, listas e sumário são gerados automaticamente pela classe.)
\begin{document}

\chapter{Introdução}
\label{ch:introducao}
\input{tex/introducao}

\chapter{Fundamentação Teórica}
\label{ch:fundamentacao}
\section{Mercado financeiro em geral} \label{sec:mercado}

Em primeiro lugar, é preciso compreender que o mercado financeiro é um espaço institucionalizado no qual diversos agentes econômicos negociam diferentes tipos de ativos, como títulos de dívida, \textit{commodities}, derivativos e moedas \cite{MishkinEakins2018}. 

% Entre esses ativos, encontram-se as ações, que representam frações do capital social de empresas de capital aberto, negociadas regularmente nas bolsas de valores\cite{Lei6404}. A posse de ações transforma investidores em acionistas, ou seja, proprietários parciais das empresas emissoras, conferindo-lhes direitos específicos, como o recebimento proporcional de lucros distribuídos sob forma de dividendos, além da participação indireta nas decisões corporativas. A valorização das ações depende tanto do desempenho financeiro da empresa quanto das expectativas futuras em relação a ela, o que torna sua negociação fortemente suscetível à percepção dos investidores e à dinâmica econômica geral.

As ações também fazem parte desses ativos, frações adequadas do capital social de empresas de capital aberto cuja negociação é realizada regularmente por meio das bolsas de valores \cite{Lei6404}. Investidores que compram uma ação tornam-se acionistas, ou proprietários parciais das corporações que emitiram tais ações, e, consequentemente, têm certos direitos legais (o direito de receber sua parte proporcional dos dividendos distribuídos pela corporação e, muitas das vezes, o direito de votar em decisões corporativas indiretamente). Isso significa que as ações não são apenas sobre como a empresa ganha dinheiro na realidade, mas também sobre o que os investidores acham que ela gerará no futuro — portanto, a negociação de ações nada mais é do que uma percepção entre as pessoas e as dinâmicas econômicas gerais.

A B3 opera em um ambiente regulado que garante segurança jurídica e transparência às operações realizadas pelos participantes, permitindo que investidores forneçam recursos às empresas ou ao governo em troca de remuneração futura, geralmente sob forma de juros, dividendos ou ganhos de capital advindos da valorização dos ativos adquiridos de acordo com a \cite{Lei6385}. 

Existem muitas bolsas de valores espalhadas por todo o mundo, cada uma operando sob regulamentações específicas, negociando ativos próprios e refletindo as particularidades econômicas e culturais das regiões onde estas estão inseridas. Exemplos notórios incluem a Bolsa de Nova York \cite{NYSE}, a Nasdaq \cite{Nasdaq} nos Estados Unidos, a Bolsa de Londres no Reino Unido \cite{LSE}, e a Bolsa de Valores de São Paulo no Brasil \cite{B3}.



\subsection{Regulação e Estrutura da B3 como Infraestrutura de Mercado}

Considerando o funcionamento do mercado financeiro brasileiro, é essencial destacar o papel central da B3 \cite{b3_quem_somos}. A B3 é reconhecida como uma das principais empresas de infraestrutura de mercado financeiro no mundo, atuando nos ambientes de bolsa e de balcão organizado \cite{b3_quem_somos}.

Neste contexto, a B3 atua como contraparte central, integrando funções de registro, custódia, liquidação e supervisão em uma cadeia verticalmente estruturada. Essa estrutura assegura a formação adequada de preços e a ampla divulgação das ofertas no livro central, com o objetivo de proteger os investidores e garantir eficiência ao mercado \cite{b3_servicos,b3_quem_somos}.

Logo, os participantes do mercado — corretoras, formadores de mercado, provedores de tecnologia e emissores — devem operar conforme os procedimentos estabelecidos pela B3, incluindo padrões de conexão como DMA, co-location e PNP, além da manutenção de sistemas de auditoria e controles internos descritos no Manual de Procedimentos Operacionais de Negociação \cite{b3_manual_negociacao}.




\subsection{Conectividade e Ambientes de Negociação}

Para que os modelos de predição sejam robustos, é essencial compreender os diferentes tipos de conexão ao ambiente de negociação da B3. Segundo o Manual de Procedimentos Operacionais de Negociação da B3, as conexões ao ambiente de negociação são classificadas conforme sua origem e hospedagem — incluindo acesso via mesa de operações, assessores e conexão direta patrocinada (DMA) \cite{b3_manual_negociacao}. Essa última, conhecida como DMA (Direct Market Access), permite que investidores ou provedores enviem ordens diretamente ao sistema de negociação, sem intermediação manual, reduzindo latência e aumentando a eficiência na execução de trades \cite{b3_manual_negociacao}.

Além do mais, a B3 oferece o serviço de \textit{co-location}, onde servidores de corretoras ou investidores são instalados fisicamente em data centers da bolsa (como COLO0/COLO1), diminuindo ainda mais a latência em relação ao DMA tradicional fora do centro de dados da B3 \cite{b3_manual_negociacao}. Essa configuração é especialmente importante para estratégias de alta frequência (HFT), nas quais milissegundos influenciam diretamente os retornos.

Portanto, ao lidar com dados de alta frequência em seu modelo de predição é fundamental considerar que os atrasos (latência) e o tipo de conexão (DMA, co-location ou assessor) impactam a qualidade e a reprodutibilidade dos resultados no ambiente real de negociação.




\subsection{Ativos}

Os ativos financeiros são instrumentos ou contratos negociados no mercado financeiro, com características e funções específicas, permitindo aos investidores diversificar seus investimentos de acordo com diferentes perfis de risco e expectativas de retorno \cite{MishkinEakins2018, FabozziModigliani2009}. Conforme introduzido na \autoref{sec:mercado}, esses ativos incluem \textbf{ações, títulos públicos e privados, moedas, derivativos e commodities} \cite{FabozziModigliani2009}. Cada categoria apresenta particularidades quanto ao comportamento dos preços, liquidez e risco associado, tornando imprescindível uma análise detalhada antes de qualquer decisão de investimento. 

Os preços desses ativos são influenciados por fatores variados, desde condições climáticas e ambientais até eventos geopolíticos e econômicos globais \cite{WorldBank2020}, o que faz deles ativos voláteis, cujos preços podem oscilar significativamente em curtos períodos. Além disso, muitos desses ativos são interligados, e variações no preço de um deles podem influenciar diretamente os demais, aumentando a complexidade das análises financeiras e exigindo dos investidores estratégias consistentes e informações atualizadas para tomadas de decisão mais assertivas \cite{MishkinEakins2018}.




\subsection{Principais conceitos utilizados no mercado financeiro}

Descrever a direção predominante do movimento dos preços ao longo do tempo caracteriza o que chamamos de \textbf{tendência}. Quando um ativo apresenta uma sequência consistente de preços ascendentes, diz-se que há uma \textbf{tendência de alta}; quando ocorre o inverso, com preços descendentes de forma persistente, trata-se de uma \textbf{tendência de baixa}. Se os preços não manifestam uma direção clara, oscilando em faixas relativamente estáveis, caracteriza-se como ausência de tendência ou \textbf{mercado lateralizado} \cite{BodieKaneMarcus2021}.

Além do conceito de tendência, outros termos importantes são frequentemente utilizados para descrever os movimentos e comportamentos observados na bolsa de valores. Um desses termos é a \textbf{volatilidade}, que representa o grau de oscilação dos preços dos ativos ao longo de um determinado período, refletindo a instabilidade ou a incerteza do mercado naquele momento \cite{MishkinEakins2018}. Quando se fala em \textbf{liquidez}, refere-se à facilidade com que um ativo pode ser convertido em dinheiro sem causar grande impacto no seu preço. Um ativo com alta liquidez pode ser negociado rapidamente, enquanto um ativo com baixa liquidez possui menos compradores e vendedores, dificultando sua negociação imediata \cite{FabozziModigliani2009}.

Outro termo amplamente utilizado é o \emph{dividendo}, que corresponde a uma parcela dos lucros que as empresas distribuem periodicamente aos acionistas como forma de remuneração pelo investimento realizado \cite{Lei6404,BodieKaneMarcus2021}.

É comum ouvir falar em \textbf{bear market} e \textbf{bull market}. O primeiro descreve períodos prolongados de queda nos preços das ações e pessimismo generalizado no mercado. Por outro lado, um \textbf{bull market} refere-se a fases caracterizadas por aumentos contínuos e generalizados dos preços dos ativos financeiros, acompanhados por otimismo dos investidores em relação ao futuro econômico \cite{BodieKaneMarcus2021}.


Por fim, \textbf{Índice de mercado} diz respeito a indicadores que medem o desempenho médio de um conjunto de ativos negociados em bolsa \cite{Daycoval}, como o Ibovespa no Brasil ou o S\&P 500 nos Estados Unidos, servindo como referência para avaliar o comportamento geral do mercado financeiro.


\subsection{Indicadores técnicos}

Além dos índices de mercado, existem também os chamados indicadores técnicos. Eles são muito usados na análise gráfica, justamente para tentar entender como os preços se comportam e, a partir disso, decidir o melhor momento de comprar ou vender um ativo. A diferença é que, enquanto os índices de mercado mostram o desempenho geral de um grupo de ações ou ativos, os indicadores técnicos olham para cada ativo separadamente. Eles funcionam com base em cálculos matemáticos feitos sobre o histórico de preços e sobre o volume de negociações, tentando transformar esses números em sinais que ajudem o investidor a agir com mais segurança.

Além dos índices de mercado tradicionais, existem outros tipos de indicadores, denominados indicadores técnicos, utilizados principalmente na análise gráfica para avaliar o comportamento dos preços e tomar decisões relacionadas à compra e venda de ativos. Diferentemente dos índices de mercado, que avaliam o desempenho geral de um conjunto específico de ativos, os indicadores técnicos são aplicados individualmente em cada ativo, com base em cálculos matemáticos realizados sobre os preços históricos e volumes de negociação \cite{Murphy1999}.

Entre os indicadores técnicos mais utilizados, destacam-se:

\begin{itemize}
    \item \textbf{Bandas de Bollinger}: indicam a volatilidade e possíveis níveis de suporte e resistência dos preços \cite{Bollinger2001};
    \item \textbf{Médias Móveis}: suavizam os dados históricos e ajudam a identificar a direção predominante dos preços \cite{Murphy1999};
    \item \textbf{Índice de Força Relativa (RSI)}: mede a força e a velocidade dos movimentos dos preços para avaliar possíveis situações de sobrecompra ou sobrevenda;
    \item \textbf{MACD (Moving Average Convergence Divergence)}: busca identificar mudanças nas tendências por meio das interações entre médias móveis.
\end{itemize}

A Figura \ref{fig:indicadores-tecnicos2} ilustra apenas como exemplo didático a aplicação prática desses indicadores em um gráfico de candles com periodicidade de 15 minutos. 
As \textbf{Médias Móveis de 9, 21 e 50 períodos} suavizam o comportamento dos preços e ajudam a identificar tendências de curto, médio e longo prazo \cite{Murphy1999}. 
Em torno das velas estão as \textbf{Bandas de Bollinger (20x2)}, que se expandem ou se contraem conforme a volatilidade do mercado, sinalizando possíveis regiões de suporte e resistência \cite{Bollinger2001}. 
Na parte inferior do gráfico, as \textbf{janelas do Índice de Força Relativa (RSI) com 9, 21 e 50 períodos} medem a força do movimento dos preços, indicando potenciais situações de sobrecompra ou sobrevenda \cite{Murphy1999}.


\begin{figure}[H] % h=aqui, t=topo, b=base, p=página de floats, !=relaxa limites
  \centering
  \includegraphics[width=0.9\linewidth]{indicadores-tecnicos2.png}
  \caption{Mini-chart de 15 minutos da ação PETR4 (Petrobras), com candles, Médias Móveis de 9, 21 e 50 períodos, Bandas de Bollinger (20x2) e RSI de 9, 21 e 50. Fonte: Yahoo Finance, adaptado pelo autor.}
  \label{fig:indicadores-tecnicos2}
\end{figure}


Esses indicadores são ferramentas essenciais para investidores e analistas técnicos, permitindo uma melhor compreensão dos movimentos de preços e auxiliando na definição de estratégias para investimentos em curto, médio ou longo prazo. \cite{Murphy1999}


\subsection{Preço real, o que acontece}

Normalmente, ao acompanharmos o mercado financeiro, prestamos atenção no preço final das ações. Aliás, é exatamente esse valor que vemos estampado nos jornais e sites especializados após o fechamento da bolsa \cite{MishkinEakins2018}. Só que, enfim, esse número por si só conta apenas parte da história do que realmente aconteceu durante o pregão. A verdade é que, ao longo do dia, acontecem inúmeras transações de compra e venda, fazendo com que o preço oscile diversas vezes—às vezes, até bruscamente—antes de chegar ao valor definitivo no encerramento do dia \cite{Murphy1999}.

Suponha-se que uma ação abra o pregão com preço elevado devido a notícias positivas divulgadas no dia anterior. Isso não garante que ela vá continuar assim durante todo o período. Na verdade, pode ocorrer o oposto: em poucas horas, influenciada por novas informações ou pela movimentação de grandes investidores, o preço pode despencar \cite{BodieKaneMarcus2021}. Por isso, apenas olhar o preço final é insuficiente para entender o comportamento completo da ação. É aí que surgem métricas mais especializadas.

O \textbf{preço de abertura}, por exemplo, revela o ponto inicial do dia e pode já sinalizar tendências imediatas devido a transações realizadas antes do pregão oficial—o chamado pre-market. Já o preço máximo alcançado durante o dia nos diz muito sobre a força compradora, sendo um sinal claro do otimismo momentâneo dos investidores \cite{Murphy1999}. Se esse preço máximo estiver muito distante do fechamento, talvez seja sinal de alerta, indicando uma possível reversão de tendência. O inverso ocorre com o preço mínimo, que reflete o maior nível de pessimismo momentâneo. Caso esteja próximo ao preço de fechamento, talvez seja sinal de recuperação da confiança dos investidores no decorrer do dia \cite{Pring2014}.

Além disso, o volume de negociação também é uma parte integrante de toda essa narrativa. Em suma, é o total transacionado em reais durante o dia. Isso pode revelar muito sobre o interesse dos investidores naquela ação.

Uma grande empresa como a Vale ou a Petrobras pode movimentar milhões, enquanto uma pequena movimenta pouco. Por exemplo, um aumento repentino no volume geralmente indica a chegada de algum evento significativo. Por outro lado, uma diminuição no volume pode mostrar o interesse diminuindo e também implicar que a tendência pode estar se revertendo de sua direção atual.

Por fim, é essencial destacar que essas métricas podem e devem ser analisadas em conjunto. Cada uma delas fornece pistas importantes, mas é a combinação de todas que permite uma leitura mais precisa e rica sobre o que realmente aconteceu com os preços durante o dia na bolsa. Na prática, é esse conjunto completo de informações que possibilita aos investidores e analistas técnicos tomarem decisões mais assertivas e também ajuda redes neurais a preverem com mais acurácia os movimentos futuros do mercado \cite{Murphy1999,BodieKaneMarcus2021}.


\subsubsection{Representação dos preços em Candlesticks}

 O gráfico \ref{fig:candlesticks} de candlestick é uma das formas mais intuitivas de representar as oscilações de preço de ativos financeiros. Cada vela informa, em um determinado intervalo de tempo, os preços de abertura, fechamento, máxima e mínima, como dito na imagem.

\begin{figure}[!htbp]
  \centering
  \includegraphics[width=0.6\linewidth]{candlesticks.png}
  \caption{Candlesticks. Fonte: \citeonline{Giacomel2016}.}
  \label{fig:candlesticks}
\end{figure}

Quando o fechamento é superior à abertura, a vela tende a ser clara, sinalizando valorização; se inferior, escura, indicando desvalorização. Essa configuração visual permite não apenas uma leitura rápida das tendências, mas também auxilia na identificação de possíveis reversões. De acordo com estudo recente de Phankokkruad e Worasuwannarak (2025), padrões como Doji e Hammer são usados como indicativos de mudança de direção nos preços e demonstram aplicabilidade em modelos preditivos baseados em redes neurais \cite{phankokkruad2025}.


\section{Estacionariedade e transformações usuais}

Mesmo quando os preços compartilham uma raiz unitária, embora os níveis de preços sejam comumente referidos como não estacionários em pesquisas e práticas financeiras, as flutuações de curto prazo podem frequentemente ser modeladas adequadamente por processos aproximadamente estacionários. É por isso que é uma tradição já estabelecida na área trabalhar com retornos e não com preços de ações simples \cite{campbell1997,tsay2010}. Neste texto, eles usam retornos logarítmicos definidos por \(r_t = \ln(P_t) - \ln(P_{t-1})\), o que torna a série temporal estacionária em relação à escala e permite composições aditivas ao longo do tempo \cite{campbell1997}.

Para avaliar a estacionaridade, podemos usar os seguintes testes complementares: o teste ADF (Augmented Dickey–Fuller), que tem como hipótese nula a existência de uma raiz unitária, e o KPSS, que assume como nula que os dados são estacionários em torno de um nível/tendência. Evita-se conclusões parciais através de uma leitura conjunta \cite{saiddickey1984}.

Para controle de escala e heterocedasticidade, são executadas transformações z-score móveis ou monotônicas (por exemplo, Box–Cox), ajustadas apenas nos dados de treinamento e depois aplicadas ao conjunto de validação/teste para evitar vazamento de dados \cite{hyndman2021fpp3}.

Em algumas séries intradiárias, também se remove a tendência do efeito sazonal da periodicidade diária (abertura/fechamento), seja por meio de dummies intradiários ou subtraindo a média por faixa de tempo, removendo variações sistemáticas que não se generalizam fora do padrão diário \cite{dacorogna2001}.


\subsection{Volatilidade e Liquidez Intradiária}

Segundo o Manual de Procedimentos Operacionais de Negociação da B3 (2019), mecanismos como a atuação de formadores de mercado, a estruturação das fases de negociação e parâmetros de lote mínimo influenciam significativamente os padrões intradiários de liquidez e volatilidade \cite{b3_manual_negociacao}. Observa-se, por exemplo, que há picos de volatilidade na abertura e no fechamento da sessão, enquanto o período intermediário costuma apresentar menor atividade.

A \emph{volatilidade intradiária} pode ser mensurada por indicadores como o desvio‐padrão de retornos em janelas móveis de 1, 5 ou 15 minutos \cite{andersen1997intraday}. Andersen \& Bollerslev (1997) demonstraram que a volatilidade não é constante ao longo do dia, concentrando-se em horários específicos e gerando efeitos de heterocedasticidade e autocorrelação nos retornos \cite{andersen1997intraday}.  

Por sua vez, a \emph{liquidez} refere-se à facilidade de negociação de um ativo sem afetar substancialmente seu preço, sendo frequentemente avaliada por medidas como:  
\begin{itemize}
  \item {Volume médio por minuto};  
  \item {Spread bid–ask}, diferença entre o maior preço que alguém está disposto a pagar por um ativo e o menor preço pelo qual alguém está disposto a vendê-lo;  
  \item {Índice de iliquidez de Amihud}, que relaciona o retorno absoluto ao volume negociado \cite{amihud2002illiquidity};  
  \item {Profundidade do livro de ofertas}, capturando o total de ordens disponíveis em níveis próximos ao melhor preço \cite{hasbrouck2007empirical}.  
\end{itemize}

Estudos empíricos mostram que períodos de baixa liquidez tendem a apresentar maior volatilidade, o que prejudica a acurácia de modelos preditivos baseados em padrões técnicos \cite{dufour2000time}. Além disso, ativos com elevada liquidez exibem maior previsibilidade em horizontes de curtíssimo prazo, justificando a seleção desses papéis como base de dados para este trabalho.

Do ponto de vista metodológico, a variação intradiária de volatilidade e liquidez introduz dois principais desafios ao modelo:  
\begin{enumerate}
  \item \textbf{Heterocedasticidade} — volatilidade variável ao longo do dia requer o uso de modelos capazes de capturar regimes de alta e baixa variância (ex.: GARCH intradiário);  
  \item \textbf{Autocorrelação temporal} — séries de duração de ordens e inter‐arrival times impactam a independência dos retornos, demandando abordagens como Autoregressive Conditional Duration \cite{engle1998acd}.  
\end{enumerate}

Dessa forma, a análise conjunta de volatilidade e liquidez intradiária fornece ao novo modelo, maior robustez e sensibilidade aos diferentes momentos da sessão, permitindo antecipar cenários de risco e otimizar as decisões de predição.


\subsection{Componente Temporal em Predição de Séries Financeiras}

Na predição de séries financeiras, o componente temporal é o elemento estruturante que vai além da simples ordem cronológica dos dados. Primeiro, efeitos sazonais e de calendário — como feriados, janelas de pré-abertura e fechamento — geram variações sistemáticas em volume e volatilidade que precisam ser modeladas explicitamente para evitar vieses sazonais \cite{hyndman_2018}. Em seguida, a escolha do tamanho da janela de análise (\textit{rolling window}) implica em um \textit{trade-off}: janelas muito longas diluem choques pontuais, enquanto janelas muito curtas amplificam ruídos e podem induzir sobreajuste \cite{hyndman_2018}.  

No gráfico da Figura \ref{fig:trade-off}, esse equilíbrio é ilustrado visualmente: à medida que aumentamos o tamanho da janela de análise, reduzimos o impacto do ruído de curto prazo, mas também perdemos a capacidade de reagir rapidamente a mudanças repentinas no mercado. Por outro lado, janelas muito curtas capturam com mais agilidade as variações recentes, porém amplificam flutuações aleatórias que podem induzir ao sobreajuste. Dessa forma, a figura evidencia o dilema central entre estabilidade e sensibilidade, que precisa ser considerado na escolha do horizonte temporal para modelagem de séries financeiras.

\begin{figure}[H] % h=aqui, t=topo, b=base, p=página de floats, !=relaxa limites
  \centering
  \includegraphics[width=0.3\linewidth]{trade-off.png}
  \caption{\textit{Trade-off} do tamanho da janela. Fonte: o autor}
  \label{fig:trade-off}
\end{figure}


Para avaliar corretamente esse \textit{trade-off} e estimar a performance out-of-sample, empregam-se técnicas de validação temporal avançada, tais como \textit{time series cross-validation} com bloqueio por períodos e backtesting walk-forward, que previnem o \textit{look-ahead bias} e fornecem indicadores realistas de acurácia \cite{bergmeir2012}. Adicionalmente, modelos de mudança de regime (\textit{regime-switching}), como o proposto por Hamilton (1989), capturam transições abruptas entre fases de mercado (e.g., bull vs. bear), ajustando a sensibilidade do algoritmo conforme o regime vigente \cite{hamilton1989switching}.  

No contexto deste trabalho, integrar todos esses elementos temporais — sazonalidades, janelas de análise otimizadas, validação avançada, detecção de regimes e embeddings de tempo — torna o modelo mais sensível tanto a movimentos rápidos quanto a evoluções estruturais, garantindo predições mais estáveis e acuradas em diferentes momentos da sessão de negociação.



% \subsection{Arquiteturas de Deep Learning aplicadas a Séries Temporais Financeiras}

% VOCÊ PRECISA EXPLICAR MELHOR SOBRE SÉRIES TEMPORAIS, ESTÁ MUITO POBRE. COLOQUE UM EXEMPLO DE UMA SÉRIE TEMPORAL VOLTADA PARA BOLSA DE VALORES!!! E A ARQUITETURA DO LSTM? COLOQUE UMA FIGURA ILUSTRANDO O FUNCIONAMENTO DO LSTM, LEMBRANDO QUE TODA FIGURA PRECISA SER REFERENCIADA NO TEXTO E EXPLICADA, MESMO QUE EM ALTO NÍVEL. O MESMO VC VAI FAZER PARA TRANSFORMER... OK? SÓ TOME CUIDADO, DE SER RASO D+, POIS TEM MAIS CONCEITOS ENVOLVIDOS EM DEEP LEARNING E SÉRIES TEMPORAIS E VC PRECISA ESCREVER DE UMA FORMA MAIS COMPLETA POSSÍVEL PARA AS PESSOAS QUE VÃO LER O SEU TRABALHO, DESDE PESSOAS LEIGAS ATÉ AS ENTENDIDAS DO ASSUNTO!!!


% A modelagem de séries temporais financeiras com aprendizado profundo avançou significativamente nos últimos anos, com o surgimento de arquiteturas especializadas capazes de lidar com dependências complexas e padrões não estacionários. Entre as abordagens mais consolidadas estão as redes recorrentes do tipo \textit{Long Short-Term Memory} (LSTM) e \textit{Gated Recurrent Unit} (GRU), desenvolvidas para capturar relações sequenciais e manter informações relevantes por longos períodos temporais \cite{cho2014learning}. Essas redes têm se mostrado eficazes na previsão de cotações, especialmente em ambientes onde os dados seguem padrões temporais marcados por autocorrelação.

% Entretanto, redes LSTM e GRU apresentam limitações quanto ao paralelismo e à eficiência computacional, além de dificuldades em capturar long-range dependencies em janelas muito extensas. Frente a essas restrições, arquiteturas baseadas em mecanismos de atenção, como o \textit{Transformer}, ganharam relevância. Introduzido por Vaswani et al. (2017), o Transformer utiliza atenção auto-regressiva para ponderar cada ponto da série com todos os demais, permitindo identificar padrões de forma paralela e com maior capacidade de generalização \cite{vaswani2017attention}.

% A aplicação de Transformers em finanças ainda é recente, mas promissora. Modelos como o \textit{Temporal Fusion Transformer} (TFT), proposto por Lim et al. (2021), combinam múltiplas fontes de entrada (temporais, categóricas e contínuas), camadas de atenção interpretáveis e mecanismos de seleção dinâmica de variáveis, tornando-se especialmente adequados para dados financeiros com variáveis exógenas e múltiplas frequências \cite{lim2021temporal}. 

% Além disso, estudos recentes mostram que, em ambientes de alta frequência e alta volatilidade — como em dados de \textit{tick-by-tick} — os Transformers superam LSTM em métricas como RMSE e \textit{directional accuracy} \cite{wu2020deep}. Isso se deve, principalmente, à capacidade do modelo de tratar longas janelas de observação sem perder estabilidade na retropropagação.

% Assim, a escolha arquitetural deve ser orientada não apenas pela natureza sequencial dos dados, mas também pelas características operacionais do mercado alvo (frequência, ruído, não linearidade), sendo o Transformer e suas variantes uma tendência crescente na modelagem de séries temporais financeiras.





\subsection{Séries Temporais e a Aplicação de Deep Learning em Finanças}

A modelagem de dados financeiros frequentemente se baseia no conceito de  séries temporais, que consistem em uma sequência de observações coletadas em intervalos de tempo regulares. Uma série temporal é mais do que uma simples lista de números. A sua característica fundamental é a dependência temporal, onde o valor de uma observação em um determinado momento pode ser influenciado por valores passados. No contexto da bolsa de valores, uma série temporal pode ser composta pelos preços de fechamento diário de uma ação, pelo volume de negociações a cada minuto ou por qualquer outra métrica que evolua ao longo do tempo.

Para ilustrar, considere a Tabela \ref{tab:exemplo_serie_temporal}, que apresenta uma série temporal multivariada hipotética para a ação PETR4, com dados intradiários coletados a cada 15 minutos. Cada linha representa um ponto no tempo, e as colunas (variáveis) representam diferentes métricas financeiras, como os preços de abertura, máxima, mínima, fechamento e o volume negociado. A análise dessa série busca identificar padrões, tendências e sazonalidades que possam ser utilizados para prever os movimentos futuros dos preços.

\begin{table}[H]
\centering
\caption{Exemplo de Série Temporal Multivariada da Ação PETR4 (intervalo de 15 minutos)}
\label{tab:exemplo_serie_temporal}
\begin{tabular}{lccccc}
\hline
\textbf{Timestamp} & \textbf{Abertura (R\$)} & \textbf{Máxima (R\$)} & \textbf{Mínima (R\$)} & \textbf{Fechamento (R\$)} & \textbf{Volume} \\
\hline
2025-08-18 10:00 & 30,50 & 30,65 & 30,48 & 30,62 & 1.250.000 \\
2025-08-18 10:15 & 30,62 & 30,70 & 30,55 & 30,68 &   890.000 \\
2025-08-18 10:30 & 30,68 & 30,69 & 30,50 & 30,52 & 1.100.000 \\
2025-08-18 10:45 & 30,52 & 30,58 & 30,45 & 30,55 &   950.000 \\
\hline
\end{tabular}
\end{table}


A modelagem de séries temporais financeiras com aprendizado profundo avançou significativamente, com o surgimento de arquiteturas especializadas capazes de lidar com dependências complexas e padrões não estacionários. A seguir, detalham-se duas das arquiteturas mais influentes neste domínio: LSTM e Transformer.

\subsubsection{Redes Recorrentes Long Short-Term Memory (LSTM)}

As Redes Neurais Recorrentes (RNNs) foram projetadas para lidar com dados sequenciais, possuindo "loops" que permitem que a informação persista ao longo do tempo. Contudo, as RNNs tradicionais enfrentam o problema do desaparecimento do gradiente (vanishing gradient), o que as impede de aprender dependências de longo prazo, uma característica essencial para séries temporais financeiras.

Para superar essa limitação, Hochreiter e Schmidhuber (1997) propuseram a arquitetura Long Short-Term Memory (LSTM), uma variante sofisticada de RNN. A principal inovação da LSTM é sua unidade de memória, composta por um estado de célula (cell state) e três portões (gates) que regulam o fluxo de informação. O estado de célula atua como uma esteira rolante, transportando informações relevantes ao longo da sequência com poucas alterações. Os portões são estruturas neurais que, por meio de funções de ativação sigmoide, decidem qual informação deve ser adicionada ou removida do estado de célula.

A Figura \ref{fig:arquitetura-lstm} ilustra o funcionamento de uma célula LSTM. Seus componentes são:

\begin{itemize}
    \item \textbf{Portão de Esquecimento (Forget Gate)}: Decide qual informação do estado de célula anterior ($C_{t-1}$) deve ser descartada. Ele analisa a entrada atual ($X_t$) e o estado oculto anterior ($h_{t-1}$) para gerar um valor entre 0 (esquecer completamente) e 1 (manter completamente).
    \item \textbf{Portão de Entrada (Input Gate)}: Determina quais novas informações serão armazenadas no estado de célula. Ele é composto por duas partes: uma camada sigmoide que decide quais valores serão atualizados e uma camada tanh que cria um vetor de novos valores candidatos.
    \item \textbf{Portão de Saída (Output Gate)}: Define qual será a saída da célula. A saída é baseada no estado da célula filtrado, que passa por uma camada tanh para normalização e, em seguida, é multiplicado pela saída da camada sigmoide do portão.
\end{itemize}

\begin{figure}[H]
    \centering
    \includegraphics[width=0.7\linewidth]{arquitetura-lstm.png}
    \caption{Arquitetura de uma célula de memória LSTM, ilustrando o estado de célula e os três portões (esquecimento, entrada e saída) que regulam o fluxo de informação. Fonte: Adaptado de \cite{colah2015lstm}.}
    \label{fig:arquitetura-lstm}
\end{figure}

Essa estrutura de portões permite que a LSTM aprenda a reter informações por longos períodos, tornando-a particularmente eficaz na previsão de cotações, onde eventos passados podem influenciar significativamente os preços futuros.

\subsubsection{Arquitetura Transformer e o Mecanismo de Atenção}

Apesar de seu sucesso, as LSTMs e outras redes recorrentes possuem uma limitação inerente: seu processamento é sequencial. Isso significa que, para processar o décimo ponto de uma série, é preciso antes ter processado os nove anteriores, o que dificulta a paralelização e a captura de dependências entre pontos muito distantes na sequência.

Para resolver essas questões, \cite{vaswani2017attention} introduziram a arquitetura Transformer, que abandona a recorrência e se baseia inteiramente no mecanismo de atenção (attention mechanism). A ideia central da atenção, especificamente da auto-atenção (self-attention), é permitir que o modelo pondere a importância de todas as outras posições da sequência de entrada ao processar uma única posição. Em vez de passar a informação sequencialmente, o Transformer pode "olhar" para toda a série temporal de uma só vez e identificar quais pontos do passado são mais relevantes para prever o futuro.

A arquitetura do Transformer, ilustrada na Figura \ref{fig:arquitetura-transformer}, é composta por dois blocos principais: o Codificador (Encoder) e o Decodificador (Decoder). Seus componentes-chave incluem:

\begin{itemize}
    \item \textbf{Auto-Atenção Multi-Cabeças (Multi-Head Self-Attention)}: Em vez de calcular a atenção uma única vez, o modelo o faz múltiplas vezes em paralelo (as "cabeças"), permitindo que ele se concentre em diferentes aspectos da sequência. Por exemplo, uma cabeça pode focar na tendência de curto prazo, enquanto outra foca em padrões sazonais de longo prazo.
    \item \textbf{Codificação Posicional (Positional Encoding)}: Como o modelo não processa os dados em ordem, a informação sobre a posição de cada ponto na sequência é adicionada aos dados de entrada. Isso garante que a ordem cronológica da série temporal seja preservada.
    \item \textbf{Camadas Feed-Forward}: Após as camadas de atenção, cada posição passa por uma rede neural feed-forward para processamento adicional.
\end{itemize}

\begin{figure}[H]
\centering
\includegraphics[width=0.5\linewidth]{arquitetura-transformer.png}
\caption{A arquitetura do modelo Transformer, destacando os blocos de codificador e decodificador, as camadas de atenção multi-cabeças e as redes feed-forward. Fonte: Adaptado de Vaswani et al. (2017) \cite{vaswani2017attention}.}
\label{fig:arquitetura-transformer}\end{figure}

A aplicação de Transformers em finanças é recente, porém bastante promissora. Modelos como o Temporal Fusion Transformer (TFT) proposto por \cite{lim2021temporal},combinam múltiplas fontes de dados e camadas de atenção interpretáveis, adequando-se a dados financeiros complexos. Estudos recentes como esse, mostram que, em ambientes de alta frequência, os Transformers podem superar as LSTMs \cite{wu2020deep}, principalmente devido à sua capacidade de modelar dependências de longo prazo de forma mais eficaz e estável .

Assim, a escolha arquitetural deve ser orientada não apenas pela natureza sequencial dos dados, mas também pelas características operacionais do mercado alvo, sendo o Transformer e suas variantes uma tendência crescente na modelagem de séries temporais financeiras.
       

\chapter{Trabalhos Relacionados}
\label{ch:trabalhos-relacionados}
\input{tex/trabalhos-relacionados} 

\chapter{Próximos passos}
\label{ch:proximos-passos}

Este capítulo detalha os procedimentos que serão adotados para desenvolver e avaliar o modelo de predição de séries temporais financeiras proposto. Os próximos passos foram estruturados em cinco etapas principais: 1) Aquisição e Descrição dos Dados; 2) Pré-processamento e Engenharia de Atributos; 3) Arquitetura dos Modelos; 4) Desenho Experimental e Treinamento; e 5) Métricas de Avaliação.

\begin{figure}[H] % h=aqui, t=topo, b=base, p=página de floats, !=relaxa limites
  \centering
  \includegraphics[height=0.5\textheight]{proximos-passos-mermaid2.png}
  \caption{Guia adotado para construção de modelo. Fonte: o autor.}
  \label{fig:proximos-passos-mermaid2}
\end{figure}

\section{Aquisição e Descrição dos Dados}

A seleção e a qualidade dos dados são determinantes para previsões intradiárias confiáveis. Para mitigar vieses comuns, adotamos três diretrizes: controle de survivorship bias, ajustes corporativos e consistência temporal. Para evitar vazamentos de informação entre janelas e avaliar desempenho de forma realista, estruturamos o histórico em partições temporais sequenciais com embargo entre treino e teste, conforme práticas recomendadas para séries financeiras \cite{lopezdeprado2018,bergmeir2012}. Além disso, respeitamos regras operacionais e fases de negociação da B3 ao longo do período analisado \cite{b3_manual_negociacao}.

\begin{itemize}
    \item \textbf{Ativos:} Selecionamos um conjunto fixo de ações líquidas negociadas na B3 (ex.: PETR4, VALE3, ITUB4), mantido constante ao longo de todo o período (2020--2025) ou, alternativamente, utilizamos a composição histórica do Ibovespa por janelas mensais, preservando a representatividade de mercado. Liquidez reduz \emph{spreads} e \emph{gaps}, diminui \emph{slippage} e tende a estabilizar métricas em \emph{backtests}.
    
    \item \textbf{Ajustes e integridade:} As séries são ajustadas por desdobramentos e \textbf{grupamentos} (\emph{reverse splits}) e por proventos (dividendos/JCP/bonificações). Conferimos fuso horário e horários de pregão; removemos barras com registros anômalos (preço/volume ausentes, zeros ou \emph{spikes} espúrios). Para reduzir efeitos de microestrutura --- como \emph{bid--ask bounce} e periodicidades intradiárias --- empregamos banda morta na rotulagem e checagens de consistência \cite{hasbrouck2007empirical,andersen1997intraday}.
    
    \item \textbf{Período e frequência:} Utilizamos barras de 15 minutos entre janeiro/2020 e julho/2025. Realizamos análise de sensibilidade em 5m e 30m para avaliar robustez à granularidade, conciliando captação da dinâmica intradiária com redução do ruído de alta frequência \cite{andersen1997intraday}.
    
    \item \textbf{Fontes:} Dados obtidos junto à B3/fornecedores com histórico consolidado e \emph{timestamping} consistente \cite{b3_servicos}.
\end{itemize}

\section{Pré-processamento e Engenharia de Atributos}

\begin{itemize}
    \item \textbf{Normalização:} Aplicamos \emph{Min--Max} (ou \emph{Z--Score} em análise de sensibilidade), ajustada exclusivamente no conjunto de treino de cada etapa do \emph{walk-forward} para evitar \emph{data leakage}. O mesmo procedimento vale para quaisquer transformações ou imputações \cite{lopezdeprado2018,bergmeir2012}.
    
    \item \textbf{Criação de atributos:} Além de OHLCV, calculamos: (i) retornos logarítmicos e variações de volume; (ii) amplitude \emph{high--low} e medidas de volatilidade móvel; (iii) indicadores consagrados como MME 9/21/50, Bandas de Bollinger e RSI, que auxiliam na captura de padrões locais e regimes de volatilidade \cite{Bollinger2001,Murphy1999}.
\end{itemize}

\section{Arquitetura dos Modelos}

Adotamos uma abordagem que combina modelos de referência e arquiteturas modernas, em ordem crescente de complexidade, viabilizando comparações consistentes. A seleção foi guiada por literatura de previsão de séries temporais e pelas particularidades da negociação eletrônica na B3, onde frequência dos dados, fases do pregão e tipos de ordens afetam a formação de preços \cite{b3_manual_negociacao,hyndman_2018,box_2015,taylor_2018,hochreiter_1997,borovykh_2017}.

\begin{itemize}
    \item \textbf{Naive / Drift (Baseline 0):} Regras simples (repetir a direção da última barra; regressão linear curta) para estabelecer piso de desempenho.
    
    \item \textbf{ARIMA (Baseline 1):} Modelo estatístico para dependências lineares; parâmetros $(p,d,q)$ identificados via ACF/PACF, seguindo o procedimento Box--Jenkins \cite{box_2015}.
    
    \item \textbf{Prophet (Baseline 2):} Referência para sazonalidades diárias/semanais; incluído de forma crítica dado o contexto intradiário de 15m \cite{taylor_2018}.
    
    \item \textbf{LSTM Puro (Baseline 3):} Duas camadas LSTM empilhadas seguidas de camada densa para previsão; adequado a dependências de longo prazo \cite{hochreiter_1997}.

    \item \textbf{Híbrido CNN--LSTM (Proposta):} Convoluções 1D extraem padrões locais na janela temporal; camadas LSTM capturam dependências de maior alcance. A saída sigmoide gera probabilidade de alta no próximo intervalo, buscando equilibrar movimentos bruscos e tendências persistentes \cite{borovykh_2017}.
\end{itemize}

\section{Desenho Experimental e Treinamento}

\begin{itemize}
    \item \textbf{Validação \emph{walk-forward}:} Dividimos o histórico em blocos sequenciais. Em cada etapa, treinamos no bloco $k$ e avaliamos em $k{+}1$, aplicando \textbf{embargo temporal} de $h$ barras (com $h$ igual ao horizonte de previsão) para reduzir dependências geradas por janelas sobrepostas/rotulagem próxima \cite{lopezdeprado2018,bergmeir2012}. O processo se repete até o fim da série.
    
    \item \textbf{Seleção de hiperparâmetros:} Em cada etapa do \emph{walk-forward}, realizamos otimização bayesiana (\emph{Optuna}) no conjunto de validação interno para janela $N$, filtros da CNN, unidades LSTM, \emph{dropout}, \emph{learning rate} etc.; o melhor conjunto é re-treinado no bloco $k$ e aferido em $k{+}1$ \cite{akiba2019}.
    
    \item \textbf{Treinamento e estabilidade:} Utilizamos AdamW, \emph{early stopping}, \emph{gradient clipping} e \emph{schedulers} (\emph{one-cycle}/cosine). Fixamos sementes e versionamos dependências para reprodutibilidade \cite{loshchilov2019}.
    
    \item \textbf{Função de custo:} \emph{Binary Cross-Entropy} para classificação probabilística da direção.
    
    \item \textbf{Ambiente:} Python 3.12; TensorFlow/Keras; Pandas/Scikit-learn.
\end{itemize}

\section{Métricas de Avaliação}

\begin{itemize}
    \item \textbf{Acurácia direcional (foco):} Métrica alinhada à decisão prática (comprar/vender), com \emph{banda morta} para amortecer ruído de microestrutura \cite{hasbrouck2007empirical}.
    
    \item \textbf{Qualidade probabilística:} \emph{Brier Score} e \emph{Log-Loss} capturam calibração e discriminação; curvas de confiabilidade (ECE) avaliam calibração global \cite{brier1950,guo2017}. Balanced Accuracy, F1-score e MCC auxiliam quando há desequilíbrio entre classes.
    
    \item \textbf{Métricas de trading:} Retorno líquido, índice de Sharpe e \emph{max drawdown} quantificam impacto após custos e \emph{slippage} \cite{sharpe1994}.
\end{itemize}



A métrica principal é a \textbf{acurácia direcional} (\emph{hit rate}) após banda morta. Para refletir qualidade probabilística e possível desbalanceamento, reportamos: Brier Score, Log-Loss, Balanced Accuracy, F1, MCC e AUC-PR; avaliamos \textbf{calibração} via curvas de confiabilidade (ECE). Para utilidade prática, executamos \emph{backtests} simples (entrada/saída) com custos de transação e \emph{slippage} explícitos, reportando retorno líquido, Sharpe, \emph{max drawdown} e \emph{profit factor}.

\subsection{Backtests e Custos de Transação}

Executamos \emph{backtests} \emph{long-only} e \emph{long/short} condicionados às probabilidades previstas e a limiares calibrados. Custos fixos e proporcionais (corretagem, emolumentos) e \emph{slippage} são descontados; também reportamos \emph{turnover} e sensibilidade a custos (análise de estresse), em conformidade com as práticas e estruturas de custos divulgadas pela B3 \cite{b3_servicos}.

\subsection{Testes de Robustez e Significância}

Comparamos a série de perdas/erros do modelo proposto contra os \emph{baselines} por meio do teste de Diebold--Mariano, avaliando significância de diferenças de acurácia direcional e Brier. Segmentamos resultados por regimes de volatilidade (calmaria vs. choques) para verificar estabilidade \cite{diebold1995}.

\subsection{Reprodutibilidade}

Todo o fluxo é versionado (dados, código e artefatos). Publicamos scripts executáveis (\texttt{prepare\_data.py}, \texttt{train.py}, \texttt{evaluate.py}, \texttt{backtest.py}) com sementes fixas e manifesto de dependências, permitindo replicação e extensão dos experimentos.

\section{Ameaças à Validade}
\begin{itemize}
    \item \textbf{Qualidade de dados:} Inconsistências de \emph{timestamp} ou \emph{spikes} podem afetar métricas; mitigamos com filtros, ajustes e checagens de integridade.
    \item \textbf{Vazamento de informação:} Evitado com ajuste exclusivamente no treino, validação \emph{walk-forward} e embargo temporal \cite{lopezdeprado2018,bergmeir2012}.
    \item \textbf{Seleção de ativos:} O viés de sobrevivência é mitigado ao fixar universo ou usar composição histórica do índice.
    \item \textbf{Sobreajuste de hiperparâmetros:} Otimização aninhada e regularização (\emph{dropout}, \emph{weight decay}) reduzem risco; relatamos sensibilidade.
\end{itemize}


\section{Cronograma PFC2}

O cronograma de execução do PFC2 foi elaborado com base nos próximos passos definidos no projeto, distribuindo as atividades em etapas mensais de setembro de 2025 a janeiro de 2026. A Tabela abaixo apresenta as principais tarefas, os períodos de realização e os entregáveis correspondentes.


% -------------------------------------------------------------------
% Cronograma de Execução do PFC2
% -------------------------------------------------------------------
\setlength{\tabcolsep}{4pt}
\newcommand{\markx}{\textbf{X}}

\begin{longtable}{
  >{\raggedright\arraybackslash}p{5.5cm}   % Coluna das etapas
  >{\centering\arraybackslash}m{1.05cm}    % Set
  >{\centering\arraybackslash}m{1.05cm}    % Out
  >{\centering\arraybackslash}m{1.05cm}    % Nov
  >{\centering\arraybackslash}m{1.05cm}    % Dez
  >{\centering\arraybackslash}m{1.05cm}    % Jan/26
  >{\raggedright\arraybackslash}p{5.5cm}}  % Entregáveis / Marcos
  \caption{Cronograma de Execução do PFC2 (Set/2025--Jan/2026)} \\
  \toprule
  \textbf{Etapa (conforme Próximos Passos)} 
    & \textbf{Set} 
    & \textbf{Out} 
    & \textbf{Nov} 
    & \textbf{Dez} 
    & \textbf{Jan/26} 
    & \textbf{Entregáveis / Marcos} \\
  \midrule
  \endfirsthead

  \toprule
  \textbf{Etapa (conforme Próximos Passos)} 
    & \textbf{Set} 
    & \textbf{Out} 
    & \textbf{Nov} 
    & \textbf{Dez} 
    & \textbf{Jan/26} 
    & \textbf{Entregáveis / Marcos} \\
  \midrule
  \endhead

  \bottomrule
  \endfoot

  % ----------------- Linhas da Tabela -----------------

  Aquisição e auditoria de dados (2020--2025, 15m); 
  ajustes corporativos; integridade temporal 
    & \markx &   &   &   &   & 
      Dataset auditado; checklist de integridade; universo de ativos definido. \\

  Pré-processamento e engenharia de atributos 
  (retornos log, vol. móvel, MME 9/21/50, RSI, Bollinger) 
    &   & \markx &   &   &   & 
      \texttt{prepare\_data.py} e \texttt{features.py} est\'aveis; dataset pronto p/ treino. \\

  Particionamento e desenho experimental 
  (\textit{walk-forward}, embargo temporal, seeds, versionamento) 
    &   & \markx &   &   &   & 
      Folds definidos; protocolo de avaliação documentado; ambiente reprodutível. \\

  Baselines: Naive/Drift, ARIMA, Prophet 
  (execução nos 1--2 primeiros blocos) 
    &   & \markx &   &   &   & 
      Relatório curto com desempenho por bloco. \\

  Modelo proposto (CNN--LSTM) e LSTM puro: 
  implementação, treino e \textit{tuning} por bloco 
    &   &   & \markx &   &   & 
      \texttt{train.py}/\texttt{evaluate.py} funcionais; logs de hiperparâmetros. \\

  Avaliação preditiva: 
  hit rate (banda morta), Brier, Log-Loss, Balanced Acc., 
  F1, MCC, AUC-PR; calibração (ECE) 
    &   &   & \markx &   &   & 
      Tabela consolidada por modelo/fold; curvas de confiabilidade. \\

  Backtests (long-only e long/short) com custos e \textit{slippage}; 
  sensibilidade a custos e \textit{turnover} 
    &   &   & \markx & \markx &   & 
      \texttt{backtest.py}; relat\'orio operacional (retorno, Sharpe, drawdown). \\

  Robustez e significância: 
  Diebold--Mariano; regimes de volatilidade; sensibilidade (5/30m) 
    &   &   &   & \markx &   & 
      Tabelas de p-valores; gráficos por regime; nota técnica. \\

  Reprodutibilidade: 
  manifesto de dependências; organização de repositório/artefatos; README executável 
    &   &   &   & \markx & \markx & 
      Repositório final organizado; scripts e experimentos replicáveis. \\

  Redação de Resultados, Discussão e Ameaças à Validade 
    &   &   &   & \markx & \markx & 
      Capítulos redigidos com figuras e tabelas; limitações descritas. \\

  Revisão ABNT (citações/refs, listas, numeração) e formatação final 
    &   &   &   &   & \markx & 
      Checklist ABNT concluído; PDF final para submissão. \\

  \textbf{Entrega e preparação para defesa} 
  (slides, roteiro, ensaio) 
    &   &   &   &   & \markx & 
      Slides prontos; ensaio cronometrado; materiais anexos. \\
\end{longtable}


\chapter{Considerações finais}
\label{ch:consideracoes-finais}
\input{tex/consideracoes-finais}

% === ELEMENTOS PÓS-TEXTUAIS =============================================


% Referências (estilo ABNT alfabético em inglês)
\bibliographystyle{packages/abntexalfenglish}
\bibliography{references}

% Glossário
\chapter{Glossário}


\begin{itemize}

\item{Gap} — Espaço no gráfico de preços entre dois períodos consecutivos, causado por variação abrupta na cotação.
\item{Slippage} — Diferença entre o preço esperado e o preço efetivo de execução de uma ordem.
\item{Spread} — Diferença entre o preço de compra (bid) e o de venda (ask) de um ativo.
\item{Survivorship bias} — Viés de análise que considera apenas casos que “sobreviveram” a um processo, ignorando os demais.
\item{Splits} – No contexto de mercados financeiros, refere-se à divisão ou agrupamento das ações de uma empresa, alterando o número de papéis em circulação sem modificar o valor total investido. 
\begin{itemize}
    \item {Stock Split (desdobramento):} aumenta a quantidade de ações e reduz proporcionalmente o preço unitário.
    \item {Reverse Split (grupamento) ou Inplits:} reduz a quantidade de ações e eleva proporcionalmente o preço unitário.
\end{itemize}
\item{Time Zone} – No contexto do mercado financeiro, refere-se ao fuso horário adotado para registrar e sincronizar horários de negociações, cotações e eventos relevantes. A correta configuração do \textit{time zone} é essencial para alinhar dados de diferentes bolsas e mercados globais, garantindo precisão na análise temporal e na execução de ordens.
\item{Candlestick} – Tipo de gráfico utilizado no mercado financeiro que representa, por meio de velas, a abertura, fechamento, máxima e mínima dos preços de um ativo em um intervalo de tempo definido.
\item{Candles} – Representações gráficas, no formato de velas (\textit{candlestick}), que mostram abertura, fechamento, máxima e mínima de preços de um ativo em um intervalo de tempo no mercado financeiro.
\item{Spikes Espúrios} – No mercado financeiro, referem-se a variações abruptas e atípicas nos preços de um ativo, geralmente causadas por erros de dados, baixa liquidez ou eventos não representativos das condições reais de mercado.
\item{Tick-by-Tick} – No mercado financeiro, refere-se a dados que registram cada alteração de preço e volume negociado de um ativo, fornecendo o histórico completo das negociações em ordem cronológica.
\item{Timestamping} – No mercado financeiro, é o processo de registrar com precisão o instante exato em que uma negociação, evento ou dado de mercado ocorre, permitindo análises temporais e sincronização de informações.
\item{APIs} – No contexto de mercado financeiro e IA, são interfaces de programação que permitem a integração e troca de dados entre sistemas, possibilitando o acesso automatizado a cotações, históricos, ordens e outros serviços em tempo real.
\item{Min-Max Scaler} – No contexto de IA aplicada ao mercado financeiro, é uma técnica de normalização que transforma os dados para um intervalo pré-definido, geralmente entre 0 e 1, preservando as proporções originais. É usada para padronizar variáveis, como preços e indicadores, antes do treinamento de modelos.
\item{Z-Score} – No contexto de IA e mercado financeiro, é uma medida de padronização que indica quantos desvios-padrão um valor está distante da média. É utilizada para identificar anomalias, como preços atípicos ou variações incomuns em séries temporais.
\item{Walk-Forward} – No contexto de IA e mercado financeiro, é uma técnica de validação em séries temporais que consiste em treinar um modelo em um intervalo de dados e testá-lo no período subsequente, avançando progressivamente no tempo para avaliar seu desempenho de forma contínua.
\item{Data Leakage} – No mercado financeiro, é o uso indevido de dados futuros no treinamento de modelos de IA, causando previsões artificialmente precisas e sem validade real.
\item{Imputation} – No mercado financeiro e IA, é o processo de preencher valores ausentes em séries temporais, garantindo consistência nos dados para análise ou treinamento de modelos.
\item{OHLCV} – No mercado financeiro, sigla para \textit{Open}, \textit{High}, \textit{Low}, \textit{Close} e \textit{Volume}, representando respectivamente abertura, máxima, mínima, fechamento e volume negociado de um ativo em um intervalo de tempo.
\item{Bayesiana} – No contexto de IA e mercado financeiro, refere-se a métodos baseados no Teorema de Bayes, que atualizam probabilidades à medida que novas informações se tornam disponíveis, auxiliando na previsão e na tomada de decisão.
\item{Dropout} – Em IA aplicada ao mercado financeiro, é uma técnica de regularização em redes neurais que desativa aleatoriamente neurônios durante o treinamento, reduzindo overfitting.
\item{Learning Rate} – Em IA aplicada ao mercado financeiro, é o parâmetro que define o tamanho dos ajustes nos pesos da rede neural a cada iteração de treinamento.
\item{Early Stopping} – Em IA para mercado financeiro, técnica que interrompe o treinamento ao detectar perda de desempenho em validação, evitando overfitting.
\item{Gradient Clipping} – Em IA para mercado financeiro, técnica que limita o valor máximo dos gradientes no treinamento, evitando instabilidade e divergência do modelo.
\item{Scheduler} – Em IA para mercado financeiro, mecanismo que ajusta automaticamente a \textit{learning rate} durante o treinamento para melhorar a convergência do modelo.
\item{Binary Cross-Entropy} – Em IA para mercado financeiro, função de perda usada em classificações binárias, medindo a diferença entre probabilidades previstas e valores reais.
\item{Drawdown} – No mercado financeiro, é a redução percentual do valor de um ativo ou portfólio desde um pico até o menor ponto subsequente.

\item{backtest} – Simulação histórica de uma estratégia sob regras e dados passados.

\item {Balanced Accuracy} - Média de sensibilidade e especificidade; robusta a classes desbalanceadas.
\item {Banda morta} - Zona neutra em torno de um limiar para ignorar microvariações e ruídos de microestrutura.
\item {Bandas de Bollinger} - Envelope ao redor de uma média móvel definido por desvios-padrão, sinalizando volatilidade e possíveis suportes/resistências.
\item {Acurácia direcional (hit rate)} - Proporção de acertos na previsão da direção (alta/baixa) do próximo movimento.
\item {Bid–ask bounce} - Oscilação do preço entre bid e ask sem mudança no valor justo, gerando ruído.
\item {Bootstrap em blocos} - Reamostragem por blocos contíguos para preservar dependência temporal em séries.
\item {Brier Score} - Erro quadrático médio entre probabilidades previstas e resultados binários observados.
\item {Co-location} - Hospedagem de servidores no data center da bolsa para reduzir latência de comunicação.
\item {Custos de transação} - Despesas e fricções (corretagem, emolumentos, spread, slippage) consideradas nos backtests.
\item {Embargo temporal} - Janela de exclusão após o treino para impedir contaminação temporal da validação/teste.
\item {F1-Score} - Média harmônica entre precisão (precision) e revocação (recall).
\item {Latência} - Atraso entre envio/recebimento de dados/ordens e sua efetiva execução/registro.
\item {Log-Loss} - Perda logarítmica usada em classificadores probabilísticos; penaliza confiança indevida.
\item {LSTM (conceito)} - Rede recorrente com memória de longo prazo para dependências temporais extensas.
\item {Microestrutura de mercado} - Aspectos de formação de preço em alta frequência (livro de ofertas, tipos de ordens, latência).
\item {Otimização Bayesiana} - Busca de hiperparâmetros guiada por modelo probabilístico do desempenho.
\item {Prophet} - Modelo aditivo com tendências e sazonalidades marcado para séries temporais.
\item {Regimes de volatilidade} - Estratificação do tempo por níveis de volatilidade para análises de robustez.
\item {Seed (semente aleatória)} - Estado inicial do gerador pseudoaleatório para reprodutibilidade.
\item {Teste de Diebold–Mariano} - Teste estatístico para comparar a acurácia preditiva de dois modelos.
\item {Transformers} - Arquitetura baseada em atenção para dependências de longo alcance sem recorrência.
\item {Turnover} - Taxa de giro da carteira (volume negociado relativo ao capital) em um período.
\item {Validação walk-forward} - Treina e testa em janelas deslizantes preservando a ordem temporal.
\item {Volatilidade} - Grau de oscilação dos preços em um período; associada à dispersão dos retornos.
\item {Índice de Sharpe} - Retorno excedente por unidade de risco (desvio-padrão dos retornos).




\end{itemize}
% \printglossaries

% Siglas
\chapter{Siglas}

\begin{itemize}

\item {API} - Interface de Programação de Aplicações; permite integração e acesso automatizado a dados/serviços.
\item {ARIMA} - Modelo AutoRegressivo Integrado de Médias Móveis para dependências lineares em séries temporais.
\item {AUC-PR} - Área sob a curva Precisão–Revocação; indicada para classes desbalanceadas.
\item {CNN} - Rede Neural Convolucional que extrai padrões locais em janelas temporais.
\item {DMA} - Direct Market Access; envio de ordens diretamente ao sistema de negociação.
\item {ECE} - Expected Calibration Error; discrepância média entre probabilidades previstas e frequências observadas.
\item {HFT} - High-Frequency Trading; estratégias sensíveis a latências de milissegundos e à microestrutura.
\item {LSTM} - Long Short-Term Memory; rede recorrente com memória de longo prazo.
\item {MCC} - Coeficiente de Correlação de Matthews (–1 a 1) para qualidade de classificação binária.
\item {MME} - Média Móvel Exponencial.
\item {OHLCV} - Abertura (Open), Máxima (High), Mínima (Low), Fechamento (Close) e Volume (Volume).
\item {RSI} - Relative Strength Index; oscilador de momento para sobrecompra/sobrevenda.

\end{itemize} 

\glsaddall
\printglossaries

\end{document}
